\section{Free Groups}

\begin{questions}
\question Given a group $G$ and a subset $X\subseteq G$, set $\langle X \rangle = \bigcap\{H\leq G \mid H\supseteq X\}$ and let $W$ be the set of elements of $G$ that can be written as words in $X^{\pm}$. Prove that $W=\langle X \rangle$.
  \begin{solution}
    Let $a,b\in W$, then $a=x_1\cdots x_m$ and $b=y_1\cdots y_n$, for some $x_i\in X^\pm$, $1\leq i\leq m$, and $y_j\in X^\pm$, $1\leq j\leq n$. Then both
    \[ ab=x_1\cdots x_my_1\cdots y_n \quad \text{and} \quad a^{-1}=x_m^{-1}\cdots x_1^{-1} \]
    are words in $X^\pm$, thus $ab\in W$ and $a^{-1}\in W$. Hence $W\leq G$. Moreover each $x\in X$ is a word in $X^\pm$, so $W\supseteq X$. It follows that $\langle X \rangle\subseteq W$.

    Suppose that $H\leq G$ satisfies $H\supseteq X$. We claim that $W\subseteq H$. Let $a\in W$, then $a=x_1\cdots x_m$ for some $x_i\in X^\pm$, $1\leq i\leq m$. For each $1\leq i\leq m$, if $x_i\in X$ then $x_i\in H$. Otherwise $x_i^{-1}\in X$, hence $x_i^{-1}\in H$, and since $H$ is a subgroup $x_i\in H$.

    As $H$ is a subgroup, it follows that the product $a=x_1\cdots x_m$ lies in $H$. Thus $W\subseteq H$, and hence $W\subseteq\langle X \rangle$. The other inclusion yields $W=\langle X \rangle$.
  \end{solution}

\question Let $G$ and $H$ be groups, and $\theta,\phi\colon G\to H$ homomorphisms, and let $X\subseteq G$ be such that $G=\langle X \rangle$. Prove that if $\theta,\phi$ agree on $X$ (i.e. $x\theta=x\phi$, $\forall x\in X$) then $\theta=\phi$ (i.e. $g\theta=g\phi$, $\forall g\in G$).
  \begin{solution}
    For each $x\in X$ we have $x^{-1}\theta = (x\theta)^{-1} = (x\phi)^{-1} = x^{-1}\phi$. Thus $\theta,\phi$ agree on $X^\pm$. Let $a=x_1\cdots x_m$, for $x_i\in X^\pm$, $1\leq i\leq m$, be a word in $X^\pm$. Then
    \[ a\theta = (x_1\cdots x_m)\theta = (x_1\theta)\cdots(x_m\theta) = (x_1\phi)\cdots(x_m\phi) = (x_1\cdots x_m)\phi = a\phi. \]
    Thus $\theta,\phi$ agree on all words in $X^\pm$. So $\theta,\phi$ agree on $\langle X \rangle=G$ by \textbf{1}. Hence $\theta=\phi$.
  \end{solution}

\question Prove directly from the definition that the infinite cyclic group $Z=\{x^n \mid n\in\Z\}$ is free of rank 1.
  \begin{solution}
    We claim $Z$ is free on the subset $\{x\}$ (and hence is free of rank 1). Let $G$ be a group, and let $\theta\colon \{x\}\to G$ be any map. Define $\theta'\colon Z\to G$ by $x^n\theta'=(x\theta)^n$ for $n\in\Z$. Clearly $x\theta'=x\theta$. Additionally $\theta'$ is a homomorphism, as
    \[ (x^mx^n)\theta' = (x^{m+n})\theta' = (x\theta)^{m+n} = (x\theta)^m(x\theta)^n = (x^m\theta')(x^n\theta'). \]
    Every element of $Z$ is a word in $\{x\}^\pm$, so by \textbf{1}, $G=\langle \{x\} \rangle$. Therefore, by \textbf{2} any homomorphism $\phi\colon Z\to G$ with $x\phi=x\theta$ is equal to $\theta'$. So $\theta'$ is the unique such homomorphism, and thus $Z$ is free on $\{x\}$.
  \end{solution}

\question (H. Smith) Let $F$ be free of rank $r$ and $G$ a group isomorphic to $F$. Show that $G$ is free of rank $r$.
  \begin{solution}
    Suppose $F$ is free on the subset $X\subseteq F$, where $\abs{X}=r$, and let $\phi\colon F\to G$ be an isomorphism. We claim that $G$ is free on the subset $Y\coloneqq X\phi\subseteq G$, and hence is free of rank $\abs{Y}=\abs{X}=r$.

    Let $H$ be a group, and let $\theta\colon Y\to H$ be a map. The isomorphism $\phi$ induces a bijection $f\colon X\to Y$. Since $F$ is free on $X$, there is a unique homomorphism $\varphi\colon F\to H$ extending the map $f\theta\colon X\to Y\to H$.

    Let $\theta'\coloneqq\phi^{-1}\varphi\colon G\to F\to H$. Then $\theta'$ is a homomorphism extending $\theta$, because whenever $y\in Y$, $y\phi^{-1}=yf^{-1}\in X$, and thus
    \[ y\theta' = y\phi^{-1}\varphi = (yf^{-1})\varphi = (yf^{-1})f\theta = y\theta.  \]
    Moreover, if $\psi\colon G\to H$ is a homomorphism extending $\theta$, then $\phi\psi\colon F\to G\to H$ extends $f\theta$, because whenever $x\in X$, $x\phi=xf\in Y$, and thus $x\phi\psi = xf\psi = xf\theta$. By the uniqueness of $\varphi$ we have
    \[ \phi\psi = \varphi \implies \psi = \phi^{-1}\varphi = \theta'. \]
    Therefore $\theta'$ is the unique homomorphism extending $\theta$. This shows $G$ is free on $Y$.
  \end{solution}

\question Given a group $G$ and a normal subgroup $N\lhd G$ with $G/N=F$ a free group, prove that $G$ has a subgroup $C$ such that
  \[ N \cap C = \{e\},\ NC = G,\ C\cong F. \]

  \begin{solution}
    Suppose $F$ is free on the subset $X\subseteq F$, and let $\pi\colon G\twoheadrightarrow G/N=F$ be the canonical projection. Note that the elements of $F$ are cosets of $N$ in $G$. By the Axiom of Choice there is a function $\theta\colon X\to G$ with $x\theta\in x$ for each $x\in X$. This implies $x\theta\pi=x$ for each $x\in X$.

    Since $F$ is free on $X$, $\theta$ extends to a homomorphism $\theta'\colon F\to G$. If $x\in X$, then
    \[ x^{-1}\theta'\pi = (x\theta)^{-1}\pi = (x\theta\pi)^{-1} = x^{-1}. \]
    Therefore $x^{-1}\theta'\in x^{-1}$ for each $x\in X$, so $x\theta'\in x$ and $x\theta'\pi=x$ for each $x\in X^\pm$. Now if $x_1,\ldots,x_m\in X^\pm$, then
    \[ (x_1\cdots x_m)\theta'\pi = (x_1\theta')\cdots(x_m\theta')\pi = (x_1\theta'\pi)\cdots(x_m\theta'\pi) = x_1\cdots x_m. \]
    Hence $x\theta'\pi=x$ for all words $x$ in $X^\pm$. Since $F$ is free on $X$, $F=\langle X \rangle$, and so by \textbf{1}, $x\theta'\pi=x$, for all $x\in F$. Let $Y\coloneqq X\theta'$, and $C\coloneqq F\theta'\leq G$. We prove that $\pi{\restriction_C}$ and $\theta'$ are two-sided inverses of each other.

    We know $\theta'\pi$ is the identity on $F$. If $c\in C$ then $c=x\theta'$ for some $x\in F$. Therefore $c\pi\theta'=x\theta'\pi\theta'=x\theta'=c$, so $\pi{\restriction_C}\theta'$ is the identity on $C$. Hence $\pi{\restriction_C}$ is an isomorphism from $C$ to $F$, and so $C\cong F$. Moreover,
    \[ N\cap C = \ker\pi\cap C = \ker\pi{\restriction_C}=\{e\}. \]
    Let $g\in G$, and set $c=g\pi\theta'$. Then $c\in C$, and
    \[ gc^{-1}\pi=(g\pi)(c\pi)^{-1}=(g\pi)(g\pi\theta'\pi)^{-1}=(g\pi)(g\pi)^{-1}=e. \]
    So $gc^{-1}\in\ker\pi=N$, hence $g\in NC$ and thus $NC=G$. This completes the proof.
  \end{solution}

\question How many reduced words are there of length $l$ in the free group of rank $r$? How many of these are cyclically reduced?
  \begin{solution}
    There are $2r$ possibilities for the first character, and $2r-1$ possibilities for each subsequent character. Hence there are $2r(2r-1)^{l-1}$ such words.
  \end{solution}

\question Show that the free abelian group of rank $r$ is isomorphic to the direct product of $r$ copies of $Z$.
  \begin{solution}
    Let $F$ be free abelian on the subset $X=\{x_1,\ldots,x_r\}$. Define a copy $Z_i=\{y_i^n \mid n\in\Z\}$ of $Z$ for each $1\leq i\leq r$. The map $x_j\mapsto(1,\ldots,y_j,\ldots,1)$ for $1\leq j\leq r$ extends uniquely to a homomorphism $\phi\colon F\to Z_1\times\cdots\times Z_r$.

    Since $x_1^{e_1}\cdots x_r^{e_r}\phi=(y_1^{e_1},\cdots,y_r^{e_r})$, it follows that $\phi$ is onto. Since $F$ is free abelian on $X$, $F=\langle X \rangle$. By \textbf{1} every element of $F$ is a word in $X^\pm$, and since $F$ is abelian every such word can be written in the form $x_1^{e_1}\cdots x_r^{e_r}$, for $e_1,\ldots,e_r\in\Z$.
z
    If $x_1^{e_1}\cdots x_r^{e_r}\phi=(1,\ldots,1)$, then $(y_1^{e_1},\ldots,y_1^{e_r})=(1,\ldots,1)$, from which it follows that $e_1=\cdots=e_r=0\implies x_1^{e_1}\cdots x_r^{e_r}=e$. Thus $\phi$ has trivial kernel, and hence is an isomorphism. Thus $F\cong Z_1\times\cdots Z_r$.
  \end{solution}

\question Let $F$ be free of rank $r$. Show that $F/F'$ is free abelian of rank $r$.
  \begin{solution}
    Suppose $F$ is free on the subset $X\subseteq F$, where $\abs{X}=r$. Let $\pi\colon F\twoheadrightarrow F/F'$ be the canonical projection, and set $Y\coloneqq X\pi$. Let $H$ be an abelian group, and let $\theta\colon Y\to H$ be any map. Then $\pi{\restriction_X}\theta\colon X\to H$ extends uniquely to $\phi\colon F\to H$.

    For any $a,b\in F$, $[a,b]\phi=[a\phi,b\phi]=e$, because $H$ is abelian. So $[a,b]\in\ker\phi$, and hence $F'\leq\ker\phi$. Thus the map $\theta'\colon F/F'\to H$ given by $x\pi\mapsto x\phi$ is a well defined homomorphism. It extends $\theta$, because for each $y\in Y$, $y=x\pi$ for some $x\in X$, and $y\theta'=x\pi\theta'=x\phi=x\pi\theta=y\theta$.

    Let $\varphi\colon F/F'\to H$ extend $\theta$, then $\pi{\restriction_X}\varphi=\pi{\restriction_X}\theta$, which implies that $\pi\varphi$ extends $\pi{\restriction_X}\theta$. By uniqueness of $\phi$, we have $\pi\varphi=\phi=\pi\theta'$. Since $\pi$ is surjective, $\varphi=\theta'$. Hence $\theta'$ is the unique homomorphism extending $\theta$, so $F/F'$ is free abelian on $Y$.

    Let $Z_i$ be as in \textbf{7}, define $H\coloneqq Z_1\times\cdots\times Z_r$, and suppose $X=\{x_1,\ldots,x_r\}$. The map $x_j\mapsto(1,\ldots,y_j,\ldots,1)$ induces a homomorphism $\phi\colon F\to H$. Because $H$ is abelian, the map $\theta'\colon F/F'\to H$ given by $x\pi\mapsto x\phi$ is a well defined homomorphism. Since $\phi=\pi\theta'$, and $\phi{\restriction_X}=\pi{\restriction_X}\theta'$ is injective, it follows that $\pi{\restriction_X}$ is injective. Thus $\abs{Y}=\abs{X}=r$, so $F/F'$ is free abelian of rank $r$.
  \end{solution}

\question Prove that the free group $F$ of finite rank $r$ has only a finite number of normal subgroups of index $n\in\N$. [\emph{Hint:} Consider homomorphisms from $F$ onto groups $G$ of order $n$.] Is the same true for arbitrary subgroups (not necessarily normal) of index $n$?
  \begin{solution}
    A given normal subgroup $N\lhd F$ of index $n$ induces a homomorphism $F\twoheadrightarrow F/N$ with kernel $N$. Since $F/N$ is a group of order $n$, the (faithful) regular action of $F/N$ on itself yields an embedding $F/N\hookrightarrow S_n$. Composing these maps produces a homomorphism $\phi\colon F\to S_n$ with $\ker\phi=N$.

    Therefore it suffices to prove that $\Mor(F,S_n)$ is finite. Suppose $F$ is free on $X$, where $\abs{X}=r$. Then $\abs{\Mor(F,S_n)}=\abs{\Map(X,S_n)}=\abs{X}^{\abs{S_n}}=r^{n!}$ is finite.
  \end{solution}

\question Let $F$ be free on a subset $X\subseteq F$, and let $Y$ be a subset of $X$. Prove that the subgroup $H=\langle Y \rangle\leq F$ is free on the set $Y$.
  \begin{solution}
    Since $Y$ generates $H$, it suffices to prove that no reduced word in $Y^\pm$ of positive length is the identity. Since $Y\subseteq X$, it follows that any reduced word in $Y^\pm$ of positive length is a reduced word in $X^\pm$ of positive length, and is thus non-trivial because $F$ is free on $X$.
  \end{solution}

\question Let $F$ be free on $\{x,y\}$ and define $a_n=x^{-n}yx^n$, $n\in\N$. Prove that no reduced word in $a_1,\ldots,a_r$ ($r\in\N$) can be equal to $e$, and deduce that the group $F_r\coloneqq\langle a_1,\ldots,a_r \rangle$ is free on these generators. Show that the group $F_0=\bigcup_{r\in\N} F_r = \langle \{x_n \mid n\in\N\} \rangle$ is free of rank $\aleph_0$.
  \begin{solution}
    A reduced word of positive length in $\{a_1,\ldots,a_r\}^\pm$ can be written in the form $a_{i_1}^{e_1}\cdots a_{i_m}^{e_m}$ for $m>0$, $e_j\in\Z\setminus\{0\}$, $i_j\in\{1,\ldots,r\}$, with $i_j\neq i_{j+1}$, by grouping adjacent terms into powers appropriately.

    It is easily verified that $a_n^e=x^{-n}y^ex^n$, hence
    \begin{align*}
      a_{i_1}^{e_1} \cdots a_{i_m}^{e_m} &= (x^{-i_1}y^{e_1}x^{i_1})\cdots(x^{-i_m}y^{e_m}x^{i_m}) \\
                                         &= x^{-i_1}y^{e_1}x^{i_1-i_2}y^{e_2}\cdots x^{i_{m-1}-i_m}y^{e_m}x^{i_m}.
    \end{align*}
    This is an alternating product of nonzero powers of $x,y$ with $2m+1>0$ terms. Therefore as a reduced word in $\{x,y\}^\pm$ it is of positive length, and so nontrivial. Since $a_1,\ldots,a_r$ generate $F_r$, it is free on these generators.

    In fact, any reduced word of positive length in $\{a_n \mid n\in\N\}^\pm$ is a reduced word in $\{a_1,\ldots,a_r\}^\pm$ for some $r\in\N$ (because it has only finitely many terms), and is thus non-trivial. Thus $F_0$ is free on the generators $\{a_n\mid n\in\N\}$, and is therefore free of rank $\abs{\{a_n\mid n\in\N\}}=\aleph_0$.
  \end{solution}

\question Prove that two elements $a,b$ in a free group $F$ are conjugate in $F$ if and only if $\check{a}$ and $\check{b}$ (as defined in (5)) are cyclic rearrangements of one another, that is, if $\check{a}=x_1\ldots x_l$, then $\check{b}=x_r\ldots x_lx_1\ldots x_{r-1}$ for some $r$, $1\leq r\leq l$.
  \begin{solution}
    Suppose $F$ is free on $X$. Recall if $a\in F$, then $\check{a}$ is a cyclically reduced word in $X^\pm$, with $a=u^{-1}\check{a}u$ for some $u\in F$. Therefore $a,b\in F$ are conjugate iff $\check{a}$ and $\check{b}$ are conjugate. Write $\check{a}=x_1\cdots x_l$ as a cyclically reduced word in $X^\pm$.

    Suppose that $\check{a}$ and $\check{b}$ are conjugate, say $\check{b}=w^{-1}\check{a}w$ for some $w\in F$. Write $w=w_1\cdots w_n$ as a reduced word in $X^\pm$. If $w_1\notin \{x_1,x_l^{-1}\}$, then the expression $\check{b}=w_n^{-1}\cdots w_1^{-1}x_1\cdots x_lw_1\cdots w_n$ is a reduced word. Since $\check{b}$ is cyclically reduced we must have $n=0$, in which case $\check{a}=\check{b}$ are cyclic rearrangements of one another.

    Otherwise, because $x_1\neq x_l^{-1}$, $w_1$ must be equal to exactly one of $x_1$ and $x_l^{-1}$. If $w_1=x_1$, let $1\leq r \leq \min\{l,n\}$ be maximal so that $w_i=x_i$ for all $1\leq i\leq r$. Then
    \[ \check{b} = w^{-1}\check{a}w =  (w_n^{-1}\cdots w_{r+1}^{-1})(x_{r+1}\cdots x_l)(x_1\cdots x_r)(w_{r+1}\cdots w_n) \]
    is a reduced word. Since $\check{b}$ is cyclically reduced, we must have $n=r$, and so $\check{b}=x_{r+1}\cdots x_lx_1\cdots x_r$ is a cyclic rearrangement of $\check{a}$. Similarly, if $w_1=x_l^{-1}$, let $1\leq r\leq \min\{l,n\}$ is as maximal so that $w_i=x_{l-i+1}^{-1}$ for all $1\leq i\leq r$. Then
    \[ \check{b} = w^{-1}\check{a}w = (w_n^{-1}\cdots w_{r+1}^{-1})(x_{l-r+1}\cdots x_l)(x_1\cdots x_{l-r})(w_{r+1}\cdots w_n) \]
    is a reduced word. Since $\check{b}$ is cyclically reduced, we must have $n=r$, and so $\check{b}=x_{l-r+1}\cdots x_lx_1\cdots x_{l-r}$ is a cyclic rearrangement of $\check{a}$. Thus in all cases, if $\check{a}$ and $\check{b}$ are conjugate, then $\check{a}$ and $\check{b}$ are cyclic rearrangements of one another.

    Conversely, suppose that $\check{a}$ and $\check{b}$ are cyclic rearrangements of one another, say $\check{b}=x_r\cdots x_lx_1\cdots x_{r-1}$ where $1\leq r\leq l$. Then
    \[ (x_1\cdots x_{r-1})^{-1}\check{a}(x_1\cdots x_{r-1}) = x_r\cdots x_lx_1\cdots x_{r-1} = \check{b}, \]
    thus $\check{a}$ and $\check{b}$ are conjugate. It follows that $a$ and $b$ are conjugate iff $\check{a}$ and $\check{b}$ are cyclic rearrangements of one another.
  \end{solution}

\question Prove that, in a free group $F$, no non-trivial element can be conjugate to its inverse. Deduce that if $H=\langle a \rangle$ is an infinite cyclic subgroup of $F$, then $N(H)=C(a)$, where $N(H)=\{w\in F \mid w^{-1}Hw=H \}$ is the normalizer of $H$ in $F$.
  \begin{solution}
    Let $a\in F$ and suppose $a=u^{-1}\check{a}u$, where $u=u_1\cdots u_n$, $\check{a}=x_1\cdots x_l$, and $a=(u_n^{-1}\cdots u_1^{-1})(x_1\cdots x_l)(u_1\cdots u_n)$ is a reduced word in $X^\pm$ of positive length. Note that $a^{-1} = u^{-1}(\check{a})^{-1}u = (u_n^{-1}\cdots u_1^{-1})(x_l^{-1}\cdots x_1^{-1})(u_1\cdots u_n)$
    and
    \[ (a^{-1})^2 = u^{-1}(\check{a})^{-2}u = (u_n^{-1}\cdots u_1^{-1})(x_l^{-1}\cdots x_1^{-1})(x_l^{-1}\cdots x_1^{-1})(u_1\cdots u_n) \]
    are both reduced. It follows that $(\check{a^{-1}})=(x_l^{-1}\cdots x_1^{-1})=(x_1\cdots x_l)^{-1}=(\check{a})^{-1}$. Now suppose that $a$ and $a^{-1}$ are conjugate. By \textbf{12}, this implies that $\check{a}$ and $(\check{a^{-1}})=(\check{a})^{-1}$ are cyclic rearrangements of one another. But then
    \[ x_r\cdots x_lx_1\cdots x_{r-1} = x_l^{-1}\cdots x_1^{-1}, \]
    for some $1\leq r\leq l$. Set $s=\lfloor \frac{l-r}{2} \rfloor$. Isolating the $s$th character of the above words gives $x_{r+s}=x_{l-s}^{-1}$. If $l-r$ is even, this gives a contradiction because $r+s=l-s$ and $F$ is torsion free. If $l-r$ is odd, we get a contradiction because $(r+s)+1=l-s$, but the above expressions are reduced words. Hence no non-trivial element of $F$ can be conjugate to its inverse.

    Let $H=\langle a \rangle$ be an infinite cyclic subgroup of $F$. Suppose that $w\in N(H)$, then $w^{-1}Hw=H$, so $w^{-1}aw=a^k$ for some $k\in\Z$. But then $w^{-1}a^iw=(w^{-1}aw)^i=a^{ik}$, and so $w^{-1}Hw\subseteq\langle a^k \rangle$. Since $H$ is infinite, we must have $k=\pm1$.

    Because $a$ is non-trivial, by the previous discussion it cannot be conjugate to $a^{-1}$. Therefore $k=1$ and $w^{-1}aw=a$, which implies that $w\in C(a)$, so $N(H)\subseteq C(a)$. Conversely, if $w\in C(a)$ then $w^{-1}a^iw=(w^{-1}aw)^i=a^i$, i.e. $w^{-1}Hw=H$ and thus $w\in N(H)$. Therefore $C(a)\subseteq N(H)$, proving that $N(H)=C(a)$.
  \end{solution}

\question Show that no word $w\neq e$ in a free group can be conjugate to a proper power of itself, namely, to $w^n$ where $n\in\Z$ and $n\neq1$.
  \begin{solution}
    Suppose that $w$ and $w^n$ are conjugate, then by \textbf{12}, $\check{w}$ and $(\check{w^n})$ are cyclic rearrangements of one another, meaning they have the same length. In \textbf{13} it was proven that $(\check{w^{-1}})=(\check{w})^{-1}$, thus $(\check{w^n})=(\check{w})^n$ for all $n\in\Z$.

    Since $\check{w}$ is cyclically reduced, $l((\check{w^n}))=l((\check{w})^n)=\abs{n}l(\check{w})$. If $l(\check{w})$ were 0, then $w$ would be trivial, hence $\abs{n}=1$. By \textbf{13}, we know that $n\neq-1$, thus $n=1$. Therefore $w\neq e$ in a free group cannot be conjugate to a proper power of itself.
  \end{solution}

\question Prove that the set of reduced words of even length in $F=F(\{x,y\})$ forms a subgroup, call it $H$. Show that $H$ is generated by the elements
  \[ a = x^2, b = xy, c = xy^{-1}. \]
  Verify that if $w$ is one of the 30 reduced words of length 2 in $\{a,b,c\}^\pm$, then the process of expressing $w$ as a reduced word in $\{x,y\}^\pm$ cannot involve the cancellation of the second letter of $a$, $b$ or $c$, nor of the first letter of $a^{-1}$, $b^{-1}$, $c^{-1}$. Thus show that if $w$ is a reduced word of length $l$ in $\{a,b,c\}^{\pm}$, then as a reduced word in $\{x,y\}^\pm$, its length is at least $l$. Deduce that $H$ is a free group of rank 3.
  \begin{solution}
    If $u,v\in F$, we have $l(u^{-1})=l(u)$ and $l(uv)=l(u)+l(v)-2r$, $r\geq0$. Therefore the set of reduced words of even length in $F$ do form a subgroup $H$. To prove that $H=\langle a,b,c \rangle$, it suffices to prove that $\langle a,b,c \rangle$ contains all words of length 2, since any word of even length is a product of words of length 2.

    We have $c^{-1}b=y^2$, and $c^{-1}a=yx$. By pre and/or post multiplying either $xy$ or $yx$ by $x^{-2}$ and $y^{-2}$ appropriately, we can see that every element of the form $x^{e_1} y^{e_2}$ or $y^{e_1}x^{e_2}$ for $e_1,e_2\in\{\pm1\}$, lies in $H$. Since the only other words of length 2 are $x^2$, $x^{-2}$, $y^2$, $y^{-2}$, it follows that $\langle a,b,c \rangle$ contains all words of length 2.

    Let $pq$ be a reduced word of length 2 in $\{a,b,c\}^\pm$. If $p=a$, then $q\neq a^{-1}$. We can see that the first letter of $q$ in all cases is not $x^{-1}$, thus the second letter of $a$ will not be cancelled when expressing this as a reduced word in $\{x,y\}^\pm$.

    Similarly, if $p=b,c$, in all cases the first letter of $q$ is never $y^{-1},y$ respectively. If instead $q=a$ (or $b,c$), then both letters of $x^2$ (or $xy,xy^{-1}$) would have to be cancelled if the second letter of $a$ (or $b,c$) are to be cancelled. Since $a,b,c,a^{-1},b^{-1},c^{-1}$ are all distinct, this never happens when $pq$ is a reduced word in $\{a,b,c\}^\pm$

    A cancellation of the first letter of $a^{-1}$, $b^{-1}$, $c^{-1}$ can be seen as equivalent to the cancellation of the second letter of $a$, $b$, $c$ respectively, by taking inverses. Thus, if $w$ is a reduced word of length $l$ in $\{a,b,c\}^\pm$, then when expressing it as a reduced word in $\{x,y\}^\pm$, at least one letter will remain in the expression from each instance of $a,b,c,a^{-1},b^{-1},c^{-1}$, and so $w$ will have length at least $l$.

    Therefore, a reduced word of positive length in $\{a,b,c\}^\pm$ is equal to a reduced word of positive length in $\{x,y\}^\pm$, and is thus non-trivial. Since $\{a,b,c\}$ generates $H$, it follows that $H$ is free on $\{a,b,c\}$, and so is free of rank 3.
  \end{solution}

\question Generalise Exercise 15 from rank 2 to rank $r$, that is, find a set of free generators for the group of reduced words of even length in $F(\{x_1,\ldots,x_r\})$.
  \begin{solution}
    Let $F\coloneqq F(\{x_1,\ldots,x_r\})$, $X\coloneqq\{x_1,\ldots,x_r\}$, and let $H$ be the set of reduced words of even length in $X^\pm$. The formulas $l(uv)=l(u)+l(v)-2r$, $r\geq0$ and $l(u^{-1})=l(u)$ prove that $H$ is a subgroup of $F$.

    Let $a=x_1^2$, and for $2\leq i\leq r$ let $b_i=x_1x_i$, $c_i=x_1x_i^{-1}$. We claim that $H$ is free on the set $Y\coloneqq\{a,b_2,c_2,\ldots,b_r,c_r\}$. From \textbf{15}, it follows that for each $2\leq i\leq r$, $\langle Y \rangle$ contains all reduced words of length 2 with letters in $\{x_1,x_i\}^\pm$. Thus if $2\leq i,j\leq r$, $i\neq j$, and $e_1,e_2\in\Z$, then $\langle Y \rangle$ contains $x_i^{e_1}x_1$ and $x_1^{-1}x_j^{e_2}$, hence it contains $x_i^{e_1}x_j^{e_2}$.

    Therefore $\langle Y \rangle$ contains all words of length 2 with letters in $X^\pm$, hence $\langle Y \rangle=H$. Next we prove that for any reduced word $w$ of length 2 in $Y^\pm$, the process of expressing $w$ as a reduced word in $X^\pm$ cannot involve the cancellation of the second letter of an element of $Y$, nor the first letter of an element of $Y^{-1}$.

    Let $pq$ be a reduced word of length 2 in $Y^\pm$. If $p=a$, then $q\in\{b_i,c_i\}^\pm$ for some $i$, and in \textbf{15} we showed that the second character of $p$ is not cancelled in this case. Similarly, if $p=b_i,c_i$, then if $q$ is to cancel the second letter of $p$, then $q$ must contain a letter $x_i^{-1}$ or $x_i$ respectively, so $q\in\{b_i,c_i\}^\pm$. In \textbf{15} we also showed that the second character of $p$ is not cancelled in this case.

    A cancellation of the first letter of an element of $Y^{-1}$ can be seen as equivalent to the cancellation of the second letter of an element of $Y$, by taking inverses. Thus if $w$ is a reduced word of length $l$ in $Y^\pm$, then when expressing it as a reduced word in $X^\pm$, at least one letter will remain in the expression from each instance of a letter in $Y^\pm$, and so $w$ will have length at least $l$.

    In particular, a reduced word of positive length in $Y^\pm$ is equal to a reduced word of positive length in $X^\pm$, and is thus non-trivial. Therefore $H$ is free on $Y$, and so is free of rank $\abs{Y}=2r-1$.
  \end{solution}

\question For a fixed $l,r\in\N$, what can you say about the group generated by the reduced words of length $l$ in the free group of rank $r$?
  \begin{solution}
    Let $F$ be free on $X=\{x_1,\ldots,x_r\}$, and let $H$ be the subgroup generated by the reduced words of length $l$. If $l=1$, then $H$ contains $X$, and thus $H=F$. If $r=1$, then $F=\langle x_1 \rangle$ is an infinite cyclic group, and the reduced words of length $l$ are $x_1^l$ and $x_1^{-l}$. Therefore $H=\langle x_1^l \rangle$ is free of rank 1, and has index $l$ in $F$.

    Suppose that $l,r>1$. We claim that $H$ contains all reduced words of length 2. Let $1\leq i,j\leq r$, $i\neq j$, $e_1,e_2\in\{\pm1\}$. Then $x_i^{e_1}x_j^{-(l-1)e_2}$, $x_j^{le_2}$, and $x_j^{(l-1)e_2}x_i^{e_1}$ have length $l$, and hence $H$ contains both
    \[ (x_i^{e_1}x_j^{-(l-1)e_2})\cdot(x_j^{le_2})=x_i^{e_1}x_j^{e_2}, \quad \text{and} \quad (x_i^{e_1}x_j^{-(l-1)e_2})\cdot(x_j^{(l-1)e_2}x_i^{e_1})=x_i^{2e_1}. \]
    Hence $H$ contains all reduced words of length 2, and thus all words of even length. If $l$ is odd, then $l=2k+1$ for some $k\geq1$. Then for each $1\leq i\leq r$, $H$ contains $x_i^l\cdot x_i^{-2k}=x_i$. Hence $X\subseteq H$, so $H=\langle X \rangle=F$ in this case.

    If $l$ is even, then the formulas $l(uv)=l(u)+l(v)-2r$, $r\geq0$, and $l(u^{-1})=l(u)$, imply that any word in the generators of $H$ has even length. By \textbf{1}, it follows that $H$ is precisely the group of reduced words of even length in $X^\pm$. This has index 2 in $F$, and \textbf{16} tells us that in this case $H$ is free of rank $2r-1$.
  \end{solution}

\question The \emph{centre} of any group $G$ is defined by
  \[ Z(G) \coloneqq \{ z\in G \mid zg=gz,\ \forall g\in G\}. \]
  Prove that the center of every non-cyclic free group is trivial.
  \begin{solution}
    By \textbf{3}, a free group of rank 1 is cyclic. Thus if $F$ is a non-cyclic free group, it must be free on a set $X$ with at least two elements $x,y$.

    If $w\neq e$ is a reduced word beginning with $x$ or $x^{-1}$, then $yw$ begins with the letter $y$, and $wy$ does not (since $y$ cannot cancel the first character of $w$). Similarly, if begins with neither $x$ nor $x^{-1}$, then $xw$ begins with the letter $x$, and $wx$ does not.

    In either case, $w\notin Z(F)$. Therefore $Z(F)=\{e\}$ is trivial.
  \end{solution}

\question Prove that no finite group can be the union of the conjugates of a proper subgroup.
  \begin{solution}
    Suppose $G$ is finite, $H<G$. Let $G$ act on $\Omega\coloneqq\{Hx \mid x\in G\}$ by right multiplication. This action is transitive because $Hx\cdot x^{-1}y=Hy$. Thus $\frac{1}{\abs{G}}\sum_{g\in G}\chi(g)=1$ by Burnside's lemma, where $\chi(g)=\abs{\{\alpha\in\Omega \mid \alpha\cdot g=\alpha\}}$.

    Note that $\chi(e)=\abs{\Omega}=\abs{G:H}>1$, since $H<G$. Hence $\chi(g)=0$ for some $g\in G$. If $g\in x^{-1}Hx$ for some $x$, then $Hx\cdot g=Hx$, which is a contradiction. Therefore the union of all conjugates of $H$ does not contain $g$, and so is not equal to $G$.
  \end{solution}

\question (J. Wiegold) In the free group $F=F(\{a,b\})$ of rank 2, let
  \[ A = \{ a_k \mid k\in\N \} \]
  be an enumeration of those reduced words that begin and end with a power of $a$. Prove that if
  \[ B = \{b^{-k}a_kb^k \mid k\in\N \}, \]
  then the subgroup $H=\langle B \rangle$ is free on $B$. Deduce that $H<F$, but $\bigcup_{w\in F} H^w=F$.
  \begin{solution}
    Let $b_k=b^{-k}a_kb^k$. A reduced word of positive length in $B^\pm$ can be written in the form $b_{i_1}^{e_1}\cdots b_{i_m}^{e_m}$ for $m>0$, $e_j\in\Z\setminus\{0\}$, $i_j\in\{1,\ldots,r\}$, with $i_j\neq i_{j+1}$, by grouping adjacent terms into powers appropriately.

    It is easy to verified that $b_n^e=b^{-n}a_n^eb^n$, hence
    \begin{align*}
      b_{i_1}^{e_1}\cdots b_{i_m}^{e_m} &= (b^{-i_1}a_{i_1}^{e_1}b^{i_1})\cdots(b^{-i_m}a_{i_m}^{e_m}b^{i_m}) \\
                                        &= b^{-i_1}a_{i_1}^{e_1}b^{i_1-i_2}a_{i_2}^{e_2}\cdots b^{i_{m-1}-i_m}a_{i_m}^{e_m}b^{i_m}.
    \end{align*}
    For each $k\in\N$, $a_k$ is a reduced word beginning and ending with a power of $a$, and thus is non-trivial. If we write $a_k=u^{-1}\check{a_k}u$ for $u=u_1\cdots u_n$, $\check{a_k}=x_1\cdots x_l$, where $a_k=(u_n^{-1}\cdots u_1^{-1})(x_1\cdots x_l)(u_1\cdots u_n)$ is reduced, then $u_n\in\{a,a^{-1}\}$, or $n=0$ in which case $x_1,x_l\in\{a,a^{-1}\}$. Note that
    \[ a_k^e=u^{-1}\check{a_k}^eu=(u_n^{-1}\cdots u_1^{-1})(x_1\cdots x_l)^e(u_1\cdots u_n) \]
    is reduced once $(x_1\cdots x_l)^e$ is expanded if $e\in\Z\setminus\{0\}$, and thus $a_k^e$ also begins and ends with a power of $a$ for each non-zero integer $e$. Hence in the above expression for $b_{i_1}^{e_1}\cdots b_{i_m}^{e_m}$, none of the powers of $b$ are cancelled when expressing it as a word in $\{a,b\}^\pm$, so as a reduced word in $\{a,b\}^\pm$ it has positive length, so is non-trivial.

    Therefore $H$ is free on $B$. Since $F$ has rank 2, and $H$ has rank $\abs{B}=\aleph_0$, we cannot have $H=F$, thus $H<F$. We prove that every non-trivial element of $F$ is conjugate to some $a_k$, and hence to some element of $B$, proving $\bigcup_{w\in F}H^w=F$.

    Let $w=x_1\cdots x_n$ be a non-trivial reduced word in $\{a,b\}^\pm$, and suppose that $x$ does not already begin and end with a power of $a$. If $x_1\in\{a\}^\pm$ and $x_n\notin\{a\}^\pm$, then $x_1wx_1^{-1}$ begins and ends with a power of $a$. Similarly if $x_1\notin\{a\}^\pm$ and $x_n\in\{a\}^\pm$, then $x_n^{-1}wx_n$ begins and ends with a power of $a$.

    Otherwise, $x_1,x_n\notin\{a\}^\pm$, and then $a^{-1}wa$ begins and ends with a power of $a$, so that in all cases $w$ is conjugate to $a_k$ for some $k\in\N$. Hence $w$ is conjugate to $b^{-k}a_kb^k\in B$ for some $k$, and so $w$ is in some conjugate of $H$. Thus $\bigcup_{w\in F}H^w=F$.
  \end{solution}
\end{questions}

%%% Local Variables:
%%% mode: latex
%%% TeX-master: "johnson"
%%% End: