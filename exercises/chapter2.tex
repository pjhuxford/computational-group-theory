\section{Schreier's Method}
\begin{questions}
\question Prove that the relation $<$ of Definition 2 satisfies (o1), (o2), (o4).
  \begin{solution}
    Let $s=x_1\cdots x_l$, $t=y_1\cdots y_m$, $u=z_1\cdots z_n$ be reduced words in $X^\pm$. Since $l\not <l$, and $x_i=x_i$ for $1\leq i\leq r$, we have not ($s<s$), thus (O1) is satisfied.

    Suppose that $s<t$ and $t<u$. Then $l\leq m\leq n$, so if either $l<m$ or $m<n$, then $l<n$, and so $s<u$. Otherwise $l=m=n$, $x_r<y_r$ for $r=\min\{i \mid x_i\neq y_i\}$, and $y_{r'}<z_{r'}$ for $r'=\min\{j \mid y_j\neq z_j\}$. Note $x_i=y_i=z_i$ for all $1\leq i<\min\{r,r'\}$

    If $\min\{r,r'\}=r$, then $x_r<y_r\leq z_r$, so that $s<u$. If $\min\{r,r'\}=r'$, then $x_{r'}\leq y_{r'}<z_{r'}$, so that $s<u$. In all cases $s<u$, so (O2) is satisfied.

    Let $S$ be a non-empty subset of $F$, and define $n=\min\{l(s) \mid s\in S\}\subseteq\N$. We inductively define a sequence $S_1,\ldots,S_n$ of non-empty subsets of $X^\pm$. Let $i\geq1$, and suppose $S_1,\ldots,S_{i-1}$ have been defined. Set $x_j=\min S_j$, for $1\leq j<i$. Define
    \[ S_i = \{ x\in X^\pm \mid x_1\cdots x_{i-1}xy_{i+1}\cdots y_n \in S\text{ is reduced, for some } y_j\in X^\pm \}. \]
    Since $S$ contains a reduced word of length $n$, it follows that $S_1$ is non-empty. For $i>1$, suppose that $S_j$ is non-empty for each $1\leq j<i$. Then the definitions $x_j=\min S_j$ make sense, and moreover for each $x\in S_{i-1}$, there exists a reduced word $x_1\cdots x_{i-2}xy_i\cdots y_n\in S$. Taking $x=x_{i-1}$ shows that $S_i$ is non-empty.

    If we finally define $x_n=\min S_n$, then we claim that the word $s=x_1\cdots x_n$ is the least element of $S$. By definition, $x_1\cdots x_{n-1}x$ is reduced and in $S$ for each $x\in S_n$. Taking $x=x_n$ shows that $s$ is a reduced word of length $n$. Let $t=y_1\cdots y_m\in S$ be a reduced word. Then $n\leq m$ by definition of $n$. If $n<m$, then $s<t$.

    Suppose that $n=m$, and that $s\neq t$. Let $r=\min\{i \mid x_i\neq y_i\}$. Then $x_r\neq y_r\in S_r$, and thus $x_r=\min S_r<y_r$, hence $s<t$. It follows that $S$ has a least element, $s$. Therefore (O4) is satisfied.
  \end{solution}

\question What is the 666th element of $F=F(\{x,y\})$ in the ordering (1)?
  \begin{solution}
    By Exercise \textbf{1.6}, for $l\geq1$ there are $4\cdot3^{l-1}$ words of length $l$ in $F$, and hence there are
    \[ 1+4(1+\cdots+3^{l-1})=1+4\cdot\frac{3^l-1}{3-1}=2\cdot3^l-1 \]
    words of length at most $l$ in $F$. Thus there are $2\cdot3^5-1=485$ words of length at most 5 in $F$. The ordering the question refers orders the elements first by length, and then lexicographically according to the ordering $x<y<x^{-1}<y^{-1}$ on $X^\pm$.

    There are $2\cdot 3^4=162$ reduced words of length 6 beginning with either $x^2$ or $xy$, as in each case there are 3 choices for each of the 4 subsequent letters. Similarly there are $2\cdot3^2=18$ reduced words of length 6 beginning with either $xy^{-1}x^2$ or $xy^{-1}xy$.

    The words described so far make up the first $485+162+18=665$ words in $F$. Therefore the 666th word in $F$ is $xy^{-1}xy^{-1}x^2$.
  \end{solution}

\question With the same $F$ and $<$, what is the second smallest element of the commutator subgroup $F'$?
  \begin{solution}
    By Exercise \textbf{1.8}, $F/F'$ is free abelian of rank 2 on $\{F'x,F'y\}$, and by Exercise \textbf{1.7}, $F/F'\cong Z\times Z$, where $Z$ is the infinite cyclic group $\{x^n \mid n\in\Z\}$, via the isomorphism determined by $F'x\mapsto(x,1)$ and $F'y\mapsto(1,x)$.

    The composite mapping $F\twoheadrightarrow F/F'\to Z\times Z$ has kernel $F'$, and is determined by $x\mapsto(x,1)$ and $y\mapsto(1,x)$. It follows that this map sends a word $w$ to $(x^{e_1},x^{e_2})$, where $e_1$ and $e_2$ are the sums of exponents of $x$-terms and $y$-terms in the word $w$.

    Therefore, elements of $F$ are reduced words which have the same number of occurrences of $x$ and $x^{-1}$ letters, and the same number of $y$ and $y^{-1}$ letters. Thus, the second smallest element of $F'$ is $xyx^{-1}y^{-1}$.
  \end{solution}

\question With the same $F$ and $<$ again, let $H=\overline{S}$ with $S=\{x^4, y^2, (xy)^2\}$. Show that $\abs{F:H}=8$ (using $D_4$) and write down a Schreier transversal for $H$ in $F$. Compile a table like Table 1, and thus obtain a basis $B$ of $H$.
  \begin{solution}
    We think of $D_4=\langle r\coloneqq(1234), s\coloneqq(12)(34) \rangle$ as a subgroup of $S_4$. The map $\theta\colon\{x,y\}\to D_4$ given by $\theta x=r$ and $\theta y=s$ extends to a homomorphism $\theta'\colon F\to D_4$. Since $r,s\in\im\theta'$, and $D_4=\langle r,s \rangle$, it follows that $\im\theta'=D_4$.

    It is simply verified that $S\subseteq\ker\theta'$, and since $\ker\theta'\lhd F$ it follows that $H=\overline{S}\leq\ker\theta'$. Therefore, applying the First Isomorphism Theorem
    \[ \abs{F:H} = \abs{F:\ker\theta'}\cdot\abs{\ker\theta':H} \geq \abs{F:\ker\theta'} = \abs{\im\theta'}= \abs{D_4} = 8. \]

    On the other hand, the elements $x'=Hx$, $y'=Hy$ generate $F/H$, have orders 4 and 2 respectively, and satisfy the relation $(x'y')^2=e\implies y'x'=x'^3y'$. Using this second relation we can write any element of $F/H$ as a word $x'^iy'^j$, and by using the orders of $x'$ and $y'$ we can ensure that $i\in\{0,1,2,3\}$ and $j\in\{0,1\}$.

    Thus, any element of $F/H$ is one of the eight cosets given by $Hx^iy^j$, where $i\in\{0,1,2,3\}$ and $y\in\{0,1\}$. Since $\abs{F:H}\geq8$, these eight cosets must be distinct, and thus $T=\{e,x,x^2,x^3,y,xy,x^2y,x^3y\}$ is a transversal for $H$ in $F$.

    Although $T$ has the Schreier property, taking the least element of each coset should lead to simpler computations for a basis $B$ of $H$. Using the relations between $x'$ and $y'$, we see that $Hx^3=Hx^{-1}$ and $Hx^3y=Hyx$. Therefore $U=\{e,x,y,x^{-1},x^2,xy,yx,x^2y\}$ is a transversal for $H$ in $F$. In fact it has the Schreier property, and thus is a Schreier transversal for $H$ in $F$.

    The table below has rows indexed by $u\in U$ and columns indexed by $z\in\{x,y\}^\pm$, and contains the generator $uz\overline{uz}^{-1}$ in the $(u,z)$ place. The identification of the element $\overline{uz}$ can be done by using the relations between $x',y'$ described earlier, and the other computations are mechanical, as they are just manipulations of words in the free group.

    \begin{center}
      \begin{tabular}{|l||p{2cm}|p{2cm}|p{2.5cm}|p{2.5cm}|}
        \hline
        & $x$ & $y$ & $x^{-1}$ & $y^{-1}$ \\
        \hline
        \hline
        $e$ & $e$ & $e$ & $e$ & $y^{-2}$ \\
        \hline
        $x$ & $e$ & $e$ & $e$ & $xy^{-2}x^{-1}$ \\
        \hline
        $y$ & $e$ & $y^2$ & $yx^{-1}y^{-1}x^{-1}$ & $e$ \\
        \hline
        $x^{-1}$ & $e$ & $x^{-1}yx^{-1}y^{-1}$ & $x^{-4}$ & $x^{-1}y^{-1}x^{-1}y^{-1}$ \\
        \hline
        $x^2$ & $x^4$ & $e$ & $e$ & $x^2y^{-2}x^{-2}$ \\
        \hline
        $xy$ & $xyxy^{-1}$ & $xy^2x^{-1}$ & $xyx^{-1}y^{-1}x^{-2}$ & $e$ \\
        \hline
        $yx$ & $yx^2y^{-1}x^{-2}$ & $yxyx$ & $e$ & $yxy^{-1}x$ \\
        \hline
        $x^2y$ & $x^2yxy^{-1}x^{-1}$ & $x^2y^2x^{-2}$ & $x^2yx^{-2}y^{-1}$ & $e$ \\
        \hline
      \end{tabular}
    \end{center}

    The table entries generate $H$, and the non-identity elements in the first two columns
    \[ B = \{x^4, y^2, yxyx, xyxy^{-1}, xy^2x^{-1}, x^2y^2x^{-2}, x^{-1}yx^{-1}y^{-1}, yx^2y^{-1}x^{-2}, x^2yxy^{-1}x^{-1} \} \]
    form a free basis of $H$.
  \end{solution}

\question Prove that the permutations $x_1=(12)$, $x_2=(23)$, $x_3=(34)$ generate the symmetric group $S_4$. Use the transversal $U=\{e,x_1\}$ of the alternating group $A_4$ to find Schreier generators for $A_4$.
  \begin{solution}
    Note that $x_1x_2x_1=(13)$, $(13)x_3(13)=(14)$, and $x_2x_3x_2=(24)$, hence $\langle x_1,x_2,x_3 \rangle$ contains all transpositions. Since the transpositions generate the symmetric group, it follows that $x_1,x_2,x_3$  generate $S_4$.

    If we set $X=\{x_1,x_2,x_3\}$, then the Schreier generators for $A_4$ are given by
    \begin{align*}
      B &= \{ ux\overline{ux}^{-1} \mid u\in U, x\in X, ux\notin U \} \\
        &= \{ x_2x_1^{-1}, x_3x_1^{-1}, x_1x_2, x_1x_3 \} \\
        &= \{ (123), (12)(34), (132) \}.
    \end{align*}
  \end{solution}

\question Convince yourself that $U=\{x^ky^l \mid k,l\in\Z\}$ is a Schreier transversal for $F'$ in $F=F(\{x,y\})$. Write down the corresponding Schreier generators for $F'$ and express the commutator $x^{-2}y^{-3}x^2y^3$ in terms of them.
  \begin{solution}
    It is clear $U$ has the Schreier property. According to Exercise \textbf{1.8} $F/F'$ is free abelian on $x'=F'x$ and $y'=F'y$, and is isomorphic to $Z\times Z$ via $x'\mapsto (x,1)$ and $y'\mapsto(1,x)$, where $Z=\{x^n \mid n\in\Z\}$ is the infinite cyclic group.

    Thus, every element of $F/F'$ can be written uniquely in the form $(x')^k(y')^l=F'x^ky^l$, for some $k,l\in\Z$. It follows that $U$ is a Schreier transversal for $F'$ in $F$. Let $u\in U$, then $u=x^ky^l$ for some $k,l\in\Z$. Note $uy=x^ky^{k+1}\in U$, however $ux=x^ky^lx\notin U$.

    Since $F/F'$ is abelian, $F'ux=F'xu$. Since $xu=x^{k+1}y^l\in U$, we have $\overline{ux}=xu$. So the set $B$ of Schreier generators for $F'$ is given by
    \[ B = \{ ux\overline{ux}^{-1} \mid u\in U \} = \{ uxu^{-1}x^{-1} \mid u\in U \} = \{ x^ky^lxy^{-l}x^{-k-1} \mid k,l\in\Z \}. \]
    Let $u_1=e$, and inductively define $u_{i+1}=\overline{u_ix_i}$, where $x_i$ is the $i$th letter in $w=x^{-2}y^{-3}x^2y^3$. Then $u_1,\ldots,u_6$ are the truncations $e,x^{-1},\ldots,x^{-2}y^{-3}$ of $w$. We have $u_7=\overline{u_6x}=x^{-1}y^{-3}$, $u_8=\overline{u_7x}=y^{-3}$, and then finally $u_9,u_{10},u_{11}$ are $y^{-2},y^{-1},e$, respectively.

    Following the proof that $B$ generates $F'$, we find the below expression of $[x^2,y^3]$
    \[ x^{-2}y^{-3}x^2y^3 = (x^{-2}y^{-3}xy^3x)(x^{-1}y^{-3}xy^3). \]
  \end{solution}

\question Prove that every subgroup of finite index in a finitely-generated group is finitely generated.
  \begin{solution}
    Let $G$ be generated by a finite set $X$, and suppose $H\leq G$ has finite index. If $U$ is a transversal for $H$ in $G$, then $U$ is finite. If $e\in U$, then the set
    \[ B=\{ux\overline{ux}^{-1} \mid u\in U, x\in X, ux\notin U\} \]
    generates $H$. Clearly $\abs{B}\leq\abs{U}\abs{X}$, so $B$ is finite. Hence $H$ is finitely generated.
  \end{solution}

\question Express each of the seven free generators in the left half of Table 1 as products of conjugates of $a=x^3$, $b=y^2$, $c=x^{-1}y^{-1}xy$.
  \begin{solution}
    The seven free generators are $x^3$, $x^{-1}yxy^{-1}$, $yxy^{-1}x^{-1}$, $y^2$, $xyx^2y^{-1}$, $xy^2x^{-1}$, and $yx^{-1}yx$. We have
    \begin{align*}
      x^3 &= a \\
      x^{-1}yxy^{-1} &= yc^{-1}y^{-1} \\
      yxy^{-1}x^{-1} &= (yx)c^{-1}(yx)^{-1} \\
      y^2 &= b \\
      xyx^2y^{-1} &= (xy)ac(xy)^{-1} \\
      xy^2x^{-1} &= xbx^{-1} \\
      yx^{-1}yx &= (ycy^{-1})(x^{-1}bx).
    \end{align*}
  \end{solution}

\question Prove that the free group of rank 2 contains a free subgroup of any given finite rank as a normal subgroup.
  \begin{solution}
    Let $F=F(\{x,y\})$, and let $C_n=\langle x \rangle$ be the cyclic group of order $n$. Define $\theta\colon\{x,y\}\to C_n$ by $x\theta=x$, $y\theta=1$. Let $\theta'\colon F\to C_n$ be the extension of $\theta$. Then $\ker\theta'\lhd F$, and since $x\in\im\theta'$, $\theta'$ is surjective. Thus $F/\ker\theta'\cong C_n$, and hence $\ker\theta'$ has index $\abs{C_n}=n$.

    From the Nielsen-Schreier Theorem, $\ker\theta'$ is free of rank $(2-1)n+1=n+1$. Thus $F$ contains normal subgroups which are free of any given finite rank $\geq2$.
  \end{solution}

\question Prove Euler's formula for planar graphs: let $\Gamma$ have $v$ vertices, $e$ edges (all straight line segments) and $f$ faces (polygonal regions bounded by edges of $\Gamma$ and containing no vertices of $\Gamma$ in their interiors). Then $v-e+f$ is equal to the number of connected components of $\Gamma$.
  \begin{solution}
    We prove the result by induction on the number of edges of $\Gamma$. If $e=0$, then $v$ is the number of connected components of $\Gamma$, and $f=0$, thus the formula holds. Suppose $e\geq1$, and that the result holds for all graphs with fewer edges.

    Suppose $\Gamma$ has $k$ connected components. Let $\Gamma'$ be the graph $\Gamma$ but with one edge removed, and suppose $\Gamma'$ has $v'$ vertices, $e'$ edges and $f'$ faces, and $k'$ connected components. Then $e'=e-1$ and $v'=v$. By hypothesis $v'-e'+f'=k'$.

    If removing the edge disconnects a connected component of $\Gamma$, then $k'=k+1$, and $f'=f$. Thus $v-(e-1)+f=k+1$, and hence $v-e+f=k$.

    Otherwise, removing the edge doesn't disconnect a component of $\Gamma$, so that $k'=k$. If the removed edge borders a face and the exterior of $\Gamma$, then that face becomes part of the exterior in $\Gamma'$. Otherwise it separates two faces, which combine into a single face in $\Gamma'$. In each case, $f'=f-1$. Hence $v-e+f=v-(e-1)+(f-1)=k$.

    It follows by induction that Euler's formula holds in all planar graphs.
  \end{solution}

\question Use the Nielsen-Schreier theorem and the result of Exercise 1.18 to give a quick proof of Theorem 1.2.
  \begin{solution}
    Let $F$ be a free group and let $w\in F\setminus\{e\}$. By the Nielsen-Schreier theorem, $C(w)\leq F$ is a free group. Clearly $w\in Z(C(w))$, thus by Exercise 1.18, $C(w)$ is a cyclic group. It must be infinite, because $e\neq w\in C(w)$, and no non-trivial power of $w$ is $e$, by the freeness of $F$.
  \end{solution}

\question Let $F=F(X)$ be an arbitrary free group, and $H\leq F$ the subgroup consisting of words of even length. Use Schreier's method to obtain a basis for $H$ (cf. Exx. 1.15, 1.16).
  \begin{solution}
    Let $x'\in X$, then $U=\{e,x'\}$ is a Schreier transversal for $H$ in $F$. For each $x\in X$, we have $\overline{ex}=x'$ and $\overline{x'x}=e$. Therefore $ex\in U$ iff $x=x'$, and $x'x\in U$ iff $x=x'^{-1}$. It follows that
    \begin{align*}
      B &= \{ ux\overline{ux}^{-1} \mid u\in U, x\in X, ux\notin U \} \\
        &= \{ xx'^{-1} \mid x\in X\setminus\{x'\} \} \cup \{ x'x \mid x\in X\setminus\{x'^{-1}\} \},
    \end{align*}
    is a basis for $H$ in $F$. By taking the inverse of some of these elements we obtain a simpler basis $\{x'x \mid x\in X^\pm\setminus\{x'^{-1}\}\}$.
  \end{solution}

\question (M. Hall, Jr.) Let $F=F(X)$ be a free group of finite rank $\abs{X}=r$, and let $n\in\N$ be a fixed natural number. Prove that $F$ contains only finitely many $n$-element Schreier sets, and that if $U$ is one of these, the number of functions $U\times X\to U$ is finite. Deduce that $F$ has only finitely many subgroups of index $n$.
  \begin{solution}
    If $U$ is a Schreier set with $n$ elements, then $U$ contains no words of length $n$ or greater. For, if $x_1\cdots x_n\in U$, then $x_1\cdots x_j\in U$ for each $0\leq j\leq n$. By Exercise 1.6, the set $S$ of words in $X^\pm$ of length $<n$ is finite. The number of $n$-element subsets of $S$ is at most $2^{\abs{S}}$, and so is also finite.

    Hence $F$ contains only finitely many $n$-element Schreier sets $U$. If $U$ is one of these, then since $U$ and $X$ are finite, there are finitely many functions $U\times X\to U$. Thus there are finitely many pairs $(U,f)$ such that $\abs{U}=n$, $f\colon U\times X\to U$, and $U$ is a Schreier transversal of $H\coloneqq\langle \{ux (f(u,x))^{-1} \mid u\in U, x\in X\} \rangle$, and $f(u,x)=\overline{ux}$.

    Let $A$ be the set of all such pairs, and let $B$ be the set of all index $n$ subgroups of $G$. Then the map $A\to B$ given by $(U,f)\mapsto \langle \{ux(f(u,x))^{-1} \mid u\in U, x\in X\} \rangle$ is surjective. For, if $H\leq F$ has index $n$ then $H$ has a Schreier transversal $U$ of size $n$, and the map $(u,x)\mapsto\overline{ux}$ satisfies $H=\langle \{ux\overline{ux}^{-1} \mid u\in U, x\in X\} \rangle$.

    Since $A$ is finite, and we have a surjective map $A\to B$, it follows that $\abs{B}$ is finite. Hence $F$ has finitely many subgroups of index $n$.
  \end{solution}

\question Use the previous exercise to show that a finitely-generated group has only finitely many subgroups of a given finite index.
  \begin{solution}
    Suppose that $G=\langle X \rangle$ is generated by a finite set $X$. Define $F\coloneqq F(X)$, and let $\pi\colon F\to G$ be the extension of the inclusion map $\theta\colon X\to G$. Suppose $H\leq G$ has finite index $n$ in $G$. $G$ acts by right multiplication on the set of cosets $G/H$. Since $\abs{G/H}=n$, this induces a homomorphism $\varphi\colon G\to S_n$.

    Let $N\coloneqq\ker\varphi$. Let $n\in N$, then $Hn=H$, and thus $n\in H$. So $N\leq H$. Let $N',H'\leq F$ be the unique subgroups such that $N'\pi=N$ and $H'\pi=H$. Then $N'\leq H'$, and in fact $N'=\ker(\pi\varphi)$. Thus $F/N'$ is isomorphic to a subgroup of $S_n$.

    Therefore $N'$ has index $\leq n!$ in $F$, and thus $H'$ has index $\leq n!$ in $F$. By \textbf{13}, $F$ contains finitely many subgroups of index $\leq n!$. Therefore $G$ contains finitely many subgroups of index $n$.
  \end{solution}

\end{questions}

%%% Local Variables:
%%% mode: latex
%%% TeX-master: "johnson"
%%% End: