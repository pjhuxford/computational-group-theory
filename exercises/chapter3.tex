\section{Nielsen's Method}
\begin{questions}
\question By expressing the transposition: $(u_1,\ldots,u_i,\ldots,u_j,\ldots,u_n)\mapsto(u_1,\ldots,u_j,\ldots,u_i,\ldots,u_n)$ as a product of six elementary Nielsen transformations, prove that any permutation of $U$ can be effected by a regular Nielsen transformation.

\question Let $\tau$ be the transformation of $U$ sending $u_i$ to $(u_i^\gamma u_j^\delta)^\varepsilon$, where $i\neq j$ and $\gamma,\delta,\varepsilon\in\{\pm1\}$, and fixing all other $u_k$. Show that $\tau$ is a regular Nielsen transformation.

\question Prove that, except in the trivial case $U=\{e\}$, (N3)$\implies$(N1).

\question Let $U=(u_1,\ldots,u_n)$ be a basis for a subgroup $H$ of $F$, and $\tau$ a regular Nielsen transformation of $U$. Prove that $U\tau$ is a basis for $H$. (Hint: for $w\in H$, let $l(w)$, $l_\tau(w)$ denote the length of $w$ as a reduced word in $U$, $U\tau$ respectively. Then prove that $l_\tau(w)\geq l(w)/k$, when $\tau$ is an elementary Nielsen transformation of type (Tk), $k=1,2$.)

\question Prove that the ordering $\ll$ in the proof of Theorem 1 is a well-ordering on the pairs $w^\pm$ in $F\setminus E$.

\question Let $H$ be the subgroup $\overline{\{x^3, y^2, x^{-1}y^{-1}xy\}}$ of $F(\{x,y\})$ studied in Chapter 2. Find Schreier generators $B$ of $H$ using the Schreier transversal $\{e,x,x^2,y,xy,x^2y\}$. Describe a Nielsen transformation $\tau$ such that $B\tau$ is $N$-reduced.

\question Let $U$ be a Schreier transversal for a subgroup $H$ of $F(X)$ such that $u$ is of minimal length in $Hu$, for all $u\in U$. Prove that the Schreier basis $B$ is $N$-reduced.

\question Let $\phi\colon F(X)\twoheadrightarrow F(Y)$ be an epimorphism of free groups with $\abs{X}$ finite. Prove that $\abs{Y}$ is finite.

\question With $\phi$ as in the previous exercise, prove that $F(X)$ has a basis $V=V_1\cup V_2$ such that $\phi$ maps $V_1$ one-to-one onto a basis for $F(Y)$ and $V_2$ into $E$. (\emph{Hint:} Let $\tau$ be a Nielsen transformation carrying $X\phi$ into $Y$, and examine the effect of $\tau$ on $X$.)

\question Prove that if a free group $F$ contains an ascending chain of subgroups
  \[ H_1\subseteq H_2\subseteq\cdots\subseteq H_n\subseteq H_{n+1}\subseteq\cdots \]
  each characteristic in the next, then $H_i=H_j$ for all $i,j\geq$ some $N\in\N$.

\question Let $\phi$ be an automorphism of $F(\{x,y\})$ and let $c=x^{-1}y^{-1}xy$. Prove that $c\phi=w^{-1}c^{\pm1}w$ for some $w\in F$.

\question Let $F$ be free of finite rank $r$. Then it was shown in deriving Corollary 7 that $\Aut F$ is generated by $r^2$ elements. Can you improve on this?

\question Prove that $x^5y^{-1}x^8$, $xyxyx$ are primitive elements of $F(\{x,y\})$, but that $x^2$, $x^{-1}y^{-1}xy$ are not.

\question Use Corollary 10 to prove that, for any set $X$, $F(X)$ is residually finite.

\question Let $H$ be a subgroup of finite index in a group $G$. Prove that $H$ is residually finite if and only if $G$ is.
\end{questions}

%%% Local Variables:
%%% mode: latex
%%% TeX-master: "johnson"
%%% End: