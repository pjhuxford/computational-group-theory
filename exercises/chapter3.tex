\section{Nielsen's Method}
\begin{questions}
\question By expressing the transposition: $(u_1,\ldots,u_i,\ldots,u_j,\ldots,u_n)\mapsto(u_1,\ldots,u_j,\ldots,u_i,\ldots,u_n)$ as a product of six elementary Nielsen transformations, prove that any permutation of $U$ can be effected by a regular Nielsen transformation.
  \begin{solution}
    We have
    \begin{align*}
      & \phantom{{}={}} (u_1,\ldots,u_i,\ldots,u_j,\ldots,u_n)(ij)(i')(ji)(j')(ij)(i') \\
      &= (u_1,\ldots,u_iu_j,\ldots,u_j,\ldots,u_n)(i')(ji)(j')(ij)(i') \\
      &= (u_1,\ldots,u_j^{-1}u_i^{-1},\ldots,u_j,\ldots,u_n)(ji)(j')(ij)(i') \\
      &= (u_1,\ldots,u_j^{-1}u_i^{-1},\ldots,u_i^{-1},\ldots,u_n)(j')(ij)(i') \\
      &= (u_1,\ldots,u_j^{-1}u_i^{-1},\ldots,u_i,\ldots,u_n)(ij)(i') \\
      &= (u_1,\ldots,u_j^{-1},\ldots,u_i,\ldots,u_n)(i') \\
      &= (u_1,\ldots,u_j,\ldots,u_i,\ldots,u_n).
    \end{align*}
    Thus the transposition $(u_1,\ldots,u_i,\ldots,u_j,\ldots,u_n)\mapsto(u_1,\ldots,u_j,\ldots,u_i,\ldots,u_n)$ can be written as the product $(ij)(i')(ji)(j')(ij)(i')$ of six elementary Nielsen transformations. Since every permutation can be written as a product of transpositions, it follows that any permutation of $U$ can be effected by a regular Nielsen transformation.
  \end{solution}

\question Let $\tau$ be the transformation of $U$ sending $u_i$ to $(u_i^\gamma u_j^\delta)^\varepsilon$, where $i\neq j$ and $\gamma,\delta,\varepsilon\in\{\pm1\}$, and fixing all other $u_k$. Show that $\tau$ is a regular Nielsen transformation.
  \begin{solution}
    Let $\tau_1,\pi_1$ be the identity Nielsen transformations, and $\sigma_1=(ij)$. Define $\tau_{-1}=(i')$, $\sigma_{-1}=(j')(ij)(j')$, and $\pi_{-1}=(i')$. Then $\tau_\gamma,\sigma_\delta,\pi_\varepsilon$ fix all $u_k$ with $k\neq i$. Moreover $\tau_\gamma$ sends $u_i$ to $u_i^{\gamma}$, $\sigma_\delta$ sends $u_i$ to $u_iu_j^{\delta}$, and $\pi_\varepsilon$ sends $u_i$ to $u_i^{\varepsilon}$. Therefore $\tau=\tau_\gamma\sigma_\delta\pi_\varepsilon$ is a regular Nielsen transformation.
  \end{solution}

\question Prove that, except in the trivial case $U=\{e\}$, (N3)$\implies$(N1).
  \begin{solution}
    If (N1) is false, then $e\in U$. Suppose then that there is an $a\in U$, $a\neq e$. Then $l(aea^{-1})=l(e)=0$, but $l(a)-l(e)+l(a^{-1})=2l(a)>0$, and $a,e,a^{-1}\in U^\pm$ with $a\neq e\neq a^{-1}$. Thus (N3) is false. Taking the contrapositive gives the result.
  \end{solution}

\question Let $U=(u_1,\ldots,u_n)$ be a basis for a subgroup $H$ of $F$, and $\tau$ a regular Nielsen transformation of $U$. Prove that $U\tau$ is a basis for $H$. (Hint: for $w\in H$, let $l(w)$, $l_\tau(w)$ denote the length of $w$ as a reduced word in $U$, $U\tau$ respectively. Then prove that $l_\tau(w)\geq l(w)/k$, when $\tau$ is an elementary Nielsen transformation of type (Tk), $k=1,2$.)

\question Prove that the ordering $\ll$ in the proof of Theorem 1 is a well-ordering on the pairs $w^\pm$ in $F\setminus E$.
  \begin{solution}
    The ordering is defined as $w_1^\pm \ll w_2^\pm$ if and only if
    \begin{enumerate}[label=(\alph*)]
    \item $l(w_1)<l(w_2)$, or these are equal and either
    \item $\min\{L(w_1),L(w_1^{-1})\}<\min\{L(w_2),L(w_2^{-1})\}$, or
    \item these mins are equal, and $\max\{L(w_1), L(w_1^{-1})\}<\max\{L(w_2), L(w_2^{-1})\}$,
    \end{enumerate}
    where $L(w)$ is the first $\lfloor (l(w)+1)/2 \rfloor$ letters of $w$. Since $l(w)=l(w^{-1})$, (a) is well-defined on the pairs $w^\pm$ in $F\setminus E$. The max and min functions are symmetric in their arguments, so conditions (b) and (c) are also well-defined on these pairs.

    Clearly, at most one of $w_1^\pm\ll w_2^\pm$ and $w_2^\pm \ll w_1^\pm$ can hold. Suppose neither hold. WLOG $\min\{L(w_1),L(w_1^{-1})\}=L(w_1)$, and $\min\{L(w_2),L(w_2^{-1})\}=L(w_2)$. It follows that $L(w_1)=L(w_2)$, and $L(w_1^{-1})=L(w_2^{-1})$. Thus the first and second halves of $w_1$ are equal to the first and second halves of $w_2$, hence $w_1=w_2$.

    Suppose that $w_1^\pm\ll w_2^\pm$ and $w_2^\pm\ll w_3^\pm$ hold. Then $l(w_1)\leq l(w_2)\leq l(w_3)$. If any one of these inequalities is strict, then $l(w_1)<l(w_3)$ and so $w_1\ll w_3$. Suppose then that $l(w_1)=l(w_2)=l(w_3)$. WLOG $\min\{L(w_i),L(w_i^{-1})\}=L(w_i)$.

    Then $L(w_1)\leq L(w_2)\leq L(w_3)$. If any one of these inequalities is strict, then $L(w_1)<L(w_3)$ and so $w_1\ll w_3$. Suppose then that $L(w_1)=L(w_2)=L(w_3)$. Then $L(w_1^{-1})<L(w_2^{-1})<L(w_3^{-1})$, so $L(w_1^{-1})<L(w_3^{-1})$, hence $w_1\ll w_3$.

    It follows that $\ll$ is a total order. Let $S$ be a non-empty subset of the pairs $w^\pm$ in $F\setminus E$. Let $S_a$ denote the subset of $S$ consisting of the pairs $w^\pm$ in $S$ with minimum length. Let $S_b$ denote the subset of $S_a$ consisting of the pairs $w^\pm$ in $S_a$, with $\min\{L(w),L(w^{-1})\}$ minimal. Finally let $S_c$ denote the subset of $S_b$ consisting of the pairs $w^\pm$ in $S_b$ with $\max\{L(w),L(w^{-1})\}$ minimal. Then $S_a,S_b,S_c\neq\emptyset$.

    It is clear that for all $w_1^\pm\in S_c$ and $w_2^\pm\in S\setminus S_c$ we have $w_1^\pm\ll w_2^\pm$. Also for all $w_1^\pm,w_2^\pm\in S_c$, neither $w_1^\pm\ll w_2^\pm$ nor $w_2^\pm\ll w_1^\pm$ holds, which implies $w_1^\pm=w_2^\pm$. Thus $S_c$ consists of a single element, which is the minimal element of $S$. Hence $\ll$ is a well-ordering.
  \end{solution}

\question Let $H$ be the subgroup $\overline{\{x^3, y^2, x^{-1}y^{-1}xy\}}$ of $F(\{x,y\})$ studied in Chapter 2. Find Schreier generators $B$ of $H$ using the Schreier transversal $\{e,x,x^2,y,xy,x^2y\}$. Describe a Nielsen transformation $\tau$ such that $B\tau$ is $N$-reduced.
  \begin{solution}
    We have that
    \begin{align*}
      B &= \{ uz\overline{uz}^{-1} \mid u\in U, z\in \{x,y\}, uz\notin U \} \\
        &= \{ x^3, yxy^{-1}x^{-1}, y^2, xyxy^{-1}x^{-2}, xy^2x^{-1}, x^2yxy^{-1}, x^2y^2x^{-2} \}.
    \end{align*}
    We think of $B$ having the ordering above. Note that (N2) fails for $B$, in particular $l((xyxy^{-1}x^{-2})x^3)=l(xyxy^{-1}x)<l(xyxy^{-1}x^{-2})$. Thus replacing $B$ by $B(41)$ fixes this issue. Still $l((x^2yxy^{-1})^{-1}x^3)=l(yx^{-1}y^{-1}x)<l(x^2yxy^{-1})$, so replacing $B(41)$ by $B(41)(6')(61)$ fixes this issue. Finally $l((x^2y^2x^{-2})x^3)=l(x^2y^2x)<l(x^2y^2x^{-2})$, so we should replace $B(41)(6')(61)$ with $B(41)(6')(61)(71)$. But still, $l((x^2y^2x)^{-1}x^3)=l(x^{-1}y^{-2}x)<l(x^2y^2x)$, so replacing what we have with $B(41)(6')(61)(71)(7')(71)$ will fix this.

    This leaves us with
    \begin{align*}
      B' &\coloneqq B(41)(6')(61)(71)(7')(71) \\
          &= (x^3, yxy^{-1}x^{-1}, y^2, xyxy^{-1}x^{-2}, xy^2x^{-1}, x^2yxy^{-1}, x^2y^2x^{-2})(41)(6')(61)(71)(7')(71) \\
          &= (x^3, yxy^{-1}x^{-1}, y^2, xyxy^{-1}x, xy^2x^{-1}, x^2yxy^{-1}, x^2y^2x^{-2})(6')(61)(71)(7')(71) \\
          &= (x^3, yxy^{-1}x^{-1}, y^2, xyxy^{-1}x, xy^2x^{-1}, yx^{-1}y^{-1}x^{-2}, x^2y^2x^{-2})(61)(71)(7')(71) \\
          &= (x^3, yxy^{-1}x^{-1}, y^2, xyxy^{-1}x, xy^2x^{-1}, yx^{-1}y^{-1}x, x^2y^2x^{-2})(71)(7')(71) \\
          &= (x^3, yxy^{-1}x^{-1}, y^2, xyxy^{-1}x, xy^2x^{-1}, yx^{-1}y^{-1}x, x^2y^2x)(7')(71) \\
          &= (x^3, yxy^{-1}x^{-1}, y^2, xyxy^{-1}x, xy^2x^{-1}, yx^{-1}y^{-1}x, x^{-1}y^{-2}x^{-2})(71) \\
          &= (x^3, yxy^{-1}x^{-1}, y^2, xyxy^{-1}x, xy^2x^{-1}, yx^{-1}y^{-1}x, x^{-1}y^{-2}x)
    \end{align*}
    We can see that $B'$ satisfies (N2), and also (N1). The only places in which (N3) can fail in $B_1$ are if we find $a,b,c\in B_1^\pm$ with $a\neq b^{-1}\neq c$, and $l(abc)\leq l(a)-l(b)+l(c)$, which can only occur if exactly half of $b$ and is cancelled in forming both $ab$ and $bc$.

    No such cancellation can occur. Thus (N3) also holds in $B'$, and thus $\tau=(41)(6')(61)(71)(7')(71)$ is the required Nielsen transformation.
  \end{solution}

\question Let $U$ be a Schreier transversal for a subgroup $H$ of $F(X)$ such that $u$ is of minimal length in $Hu$, for all $u\in U$. Prove that the Schreier basis $B$ is $N$-reduced.

\question Let $\phi\colon F(X)\twoheadrightarrow F(Y)$ be an epimorphism of free groups with $\abs{X}$ finite. Prove that $\abs{Y}$ is finite.

\question With $\phi$ as in the previous exercise, prove that $F(X)$ has a basis $V=V_1\cup V_2$ such that $\phi$ maps $V_1$ one-to-one onto a basis for $F(Y)$ and $V_2$ into $E$. (\emph{Hint:} Let $\tau$ be a Nielsen transformation carrying $X\phi$ into $Y$, and examine the effect of $\tau$ on $X$.)

\question Prove that if a free group $F$ contains an ascending chain of subgroups
  \[ H_1\subseteq H_2\subseteq\cdots\subseteq H_n\subseteq H_{n+1}\subseteq\cdots \]
  each characteristic in the next, then $H_i=H_j$ for all $i,j\geq$ some $N\in\N$.

\question Let $\phi$ be an automorphism of $F(\{x,y\})$ and let $c=x^{-1}y^{-1}xy$. Prove that $c\phi=w^{-1}c^{\pm1}w$ for some $w\in F$.

\question Let $F$ be free of finite rank $r$. Then it was shown in deriving Corollary 7 that $\Aut F$ is generated by $r^2$ elements. Can you improve on this?

\question Prove that $x^5y^{-1}x^8$, $xyxyx$ are primitive elements of $F(\{x,y\})$, but that $x^2$, $x^{-1}y^{-1}xy$ are not.

\question Use Corollary 10 to prove that, for any set $X$, $F(X)$ is residually finite.

\question Let $H$ be a subgroup of finite index in a group $G$. Prove that $H$ is residually finite if and only if $G$ is.
\end{questions}

%%% Local Variables:
%%% mode: latex
%%% TeX-master: "johnson"
%%% End: