\section{Free Presentations of Groups}
\begin{questions}
\question Give an alternative proof of Proposition 1 by showing that if $G$ is any group, then $G$ has the presentation $\langle X \mid M \rangle$, where $X=G$ and
  \[ M = \{xy = (x,y)m \mid x,y\in G\} \]
  is the multiplication table of $G$ (where $(x,y)m$ stands in the $(x,y)$-place).
  \begin{solution}
    Let $x\in G$, and suppose that $y$ is the inverse of $x$ in $G$. Then in $\langle X\mid M \rangle$, $xy=(x,y)m=e$, so $y$ is also the inverse of $x$ in $\langle X\mid M \rangle$. From this, and by repeatedly using the relation $xy=(x,y)m$, every element of $\langle X \mid M \rangle$ is equal to a single element $x\in X$. Thus $\abs{\langle X \mid M \rangle}\leq\abs{G}$.

    Let $\theta\colon X\to G$ be the identity map. Then $(x\theta)(y\theta)=xy=(x,y)m=(x,y)m\theta$. The substitution test shows that $\theta$ extends to a homomorphism $\theta''\colon\langle X\mid M \rangle\to G$, and this is an epimorphism because $\langle X\theta \rangle=\langle G \rangle=G$. Since $\abs{\langle X\mid M \rangle}\leq\abs{G}$ are finite, $\theta''$ is an isomorphism. Hence $G$ has the finite presentation $\langle X\mid M \rangle$.
  \end{solution}

\question Let $G=\langle X \mid R \rangle$ and $G'$ be the commutator subgroup of $G$. Prove that $G_{ab}=G/G'$ has the presentation
  \[ \langle X \mid R, [X,X] \rangle, \]
  where $[X,X]=\langle [x,y] \mid x,y\in X,\ x\neq y \rangle$. Use Proposition 3 to confirm the result of Exercise 1.7.
  \begin{solution}
    Let $\pi\colon G\to G_{ab}$ be the natural map. Let $\theta=\pi{\restriction_X}$. Then for all $r\in R$, if we replace each $x\in X$ in $r$ with $x\theta=x\pi$, then we obtain $r\pi=e\pi=e$. Also, if $x,y\in X$, $x\neq y$, then $[x\theta,y\theta]=e$, since $G_{ab}$ is abelian.

    By the substitution test, $\theta$ extends to a homomorphism $\theta''\colon\langle X\mid R, [X,X] \rangle\to G_{ab}$. This is an epimorphism, since $\pi$ is surjective and $\langle X\theta \rangle=\langle X\pi \rangle=\langle X \rangle\pi=G_{ab}$.

    In $\langle X\mid R, [X,X] \rangle$, the generators all commute, thus it is an abelian group. By von Dyck's theorem, there is a homomorphism $\phi\colon G\to \langle X \mid R, [X,X] \rangle$ fixing all $x\in X$. Then $\ker\pi=G'\leq\ker\phi$, since $[x,y]\phi=[x\phi,y\phi]=e$ for all $x,y\in G$.

    Thus there is a well-defined homomorphism $\varphi\colon G_{ab}\to\langle X \mid R, [X,X] \rangle$, with $\pi\varphi=\phi$. Thus $\varphi$ maps $x\pi\mapsto x$ for $x\in X$. Additionally $\theta''$ maps $x\in X$ to $x\pi$. Since $X\pi$ generates $G_{ab}$, and $X$ generates $\langle X \mid R, [X,X] \rangle$, it follows that $\varphi$ is a two-sided inverse of $\theta''$. Hence $G_{ab}\cong\langle X \mid R,[X,X] \rangle$, as desired.
  \end{solution}

\question For $l,m,n\in\Z$, prove that
  \[ D(l,m,n) \cong D(m,l,n) \cong D(-l,m,n), \]
  and deduce that $D(l,m,n)$ is independent of the orders and signs of $l,m,n$.
  \begin{solution}
    Note that $(xy)^n=x(yx)^nx^{-1}$, thus $(xy)^n=e$ iff $(yx)^n=e$. Hence $(yx)^n\in\overline{\{x^l,y^m,(xy)^n\}}$, and $(xy)^n\in\overline{\{x^l,y^m,(yx)^n\}}$. Applying Tietze transformations $R+$ and $R-$ yields
    \[ \langle x,y \mid x^l, y^m, (xy)^n \rangle \cong \langle y,x \mid y^m, x^l, (xy)^n, (yx)^n \rangle \cong \langle y,x \mid y^m, x^l, (yx)^n \rangle, \]
    thus $D(l,m,n)\cong D(m,l,n)$. We already know that $D(l,m,n)\cong D(n,m,l)$, hence $D(l,m,n)$ is independent of the order of $l,m,n$. Similarly, $x^l=e$ iff $x^{-l}=e$, thus $x^{-l}\in\overline{\{x^l,y^m,(xy)^n\}}$ and $x^l\in\overline{\{x^{-l},y^m,(xy)^n\}}$. Applying Tietze transformations $R+$ and $R-$ yields
    \[ \langle x,y \mid x^l, y^m, (xy)^n \rangle \cong \langle x,y \mid x^l, x^{-l}, y^m, (xy)^n \rangle \cong \langle x,y \mid x^{-1}, y^m, (xy)^n \rangle, \]
    thus $D(l,m,n)\cong D(-l,m,n)$. Since $D(l,m,n)$ is independent of the order of $l,m,n$, it follows that it is also independent of the signs of $l,m,n$.
  \end{solution}

\question Identify $D(l,m,n)$ in the following cases:
  \begin{parts}
  \part one of $l,m,n$ is 1,
    \begin{solution}
      By \textbf{3}, it suffices to identify $D(m,1,n)$. Easy Tietze transformations yield
      \begin{align*}
        D(m,1,n) &= \langle x,y \mid x^m, y, (xy)^n \rangle \\
                 &\cong \langle x,y \mid x^m, y, (xy)^n, x^n \rangle \\
                 &\cong \langle x,y \mid x^m, y, x^n \rangle \\
                 &\cong \langle x \mid x^m, x^n \rangle.
      \end{align*}
      Let $d=\gcd(m,n)$. There exist integers $a,b$ such that $am+bn=d$. Hence if $x^m=e$ and $x^n=e$, then $x^d=x^{am+bn}=(x^m)^a(x^n)^b=e$. Conversely, if $x^d=e$, then since $d\mid m$, and $d\mid n$, $x^m=x^n=e$. Appropriate Tietze transformations show that
      \[ \langle x \mid x^m, x^n \rangle \cong \langle x \mid x^m, x^n, x^d \rangle \cong \langle x \mid x^n, x^d \rangle \cong \langle x \mid x^d \rangle. \]
      Thus $D(m,1,n)\cong\langle x\mid x^{\gcd(m,n)} \rangle = Z_{\gcd(m,n)}$.
    \end{solution}
  \part two of $l,m,n$ are 2.
    \begin{solution}
      By \textbf{3}, it suffices to identify $D(n,2,2)=\langle x,y \mid x^n, y^2, (xy)^2 \rangle$. In this group, since $y^2=e$, $y=y^{-1}$. From $(xy)^2=e$ we have $xyxy=e$, hence $x^y=x^{-1}$. Since conjugation by $y$ is an automorphism, it follows that $(x^i)^y=(x^y)^i=x^{-i}$, and thus $yx^i=x^{-i}y$ for all $i\in\Z$.

      This rule can be used to write any word in $\{x,y\}^\pm$ in the form $x^iy^j$, by moving all powers of $x$ to the left of any instances of $y$. Since $x^n=y^2=e$, we can also ensure that $0\leq i<n$ and $0\leq j<2$. Hence $\abs{D(n,2,2)}\leq 2n$.

      Let $D_n$ denote the dihedral group of order $2n$. If $r,s\in D_n$ denote the smallest positive rotation, and a reflection respectively, then $D_n=\langle r,s \rangle$. Moreover $r^n=s^2=e$. It can be seen geometrically that $rs$ is also a reflection, hence $(rs)^2=e$. Thus the map $\theta\colon\{x,y\}\to D_n$ given by $x\theta=r$ and $y\theta=s$ by the substitution test extends to a homomorphism $\theta''\colon D(n,2,2)\to D_n$.

      In fact, $\theta''$ is an epimorphism, since $\langle \theta\{x,y\} \rangle=\langle r,s \rangle=D_n$. Since $\abs{D(n,2,2)}\leq\abs{D_n}$, $\theta''$ must be an isomorphism. Hence $D(n,2,2)$ is isomorphic to the dihedral group of order $2n$.
    \end{solution}
  \end{parts}

\question Let $D_\infty=\langle a,b \mid a^2, b^2 \rangle$. Identify $D_\infty/D_\infty'$. Prove that $D_\infty'$ is a) cyclic, and b) infinite. (Hint: For a), consider conjugates of $[a,b]$ and for b), consider homomorphisms from $D_\infty$ onto the groups $D(2,2,n)$.)
  \begin{solution}
    Note the identities $[y,x]^{-1}=[x,y]$.
    \[ [x,zy]=x^{-1}y^{-1}z^{-1}xzy=(x^{-1}y^{-1}xy)(y^{-1}(x^{-1}z^{-1}xz)y)=[x,y][x,z]^y. \]
    Since conjugation is an automorphism, it follows that every commutator in $D_\infty$ are a product of conjugates of commutators $[x,y]$, with $x,y\in\{a,b\}^\pm=\{a,b\}$.

    Note that $[b,a]=[a,b]^a=[a,b]^b=baba=[a,b]^{-1}$, and $[a,a]=[b,b]=e$. It follows that every commutator in $D_\infty$ is a power of $[a,b]$, and thus $D_\infty'=\langle [a,b] \rangle$ is cyclic.

    By von Dyck's theorem there is an epimorphism $\phi\colon D_\infty\to D(2,2,n)$ with $a\phi=x$ and $b\phi=y$. Then $[a,b]\phi=[x,y]=(xy)^2$. By \textbf{4b}, $D(2,2,n)$ is isomorphic to the dihedral group $D_n$, via an isomorphism which maps $xy$ to a rotation of order $n$.

    Thus $(xy)^2$ has order at least $n/2$ in $D(2,2,n)$, and therefore $[a,b]$ has order at least $n/2$ in $D_\infty$ for every $n\in\N$. Hence $\abs{[a,b]}$, and thus $D_\infty'=\langle [a,b] \rangle$ is infinite.
  \end{solution}

\question Show that the groups
  \[ \langle a,b \mid a^bb^a = (b^{-1}a^2)^2 = e \rangle, \quad \langle x,y \mid x^2 = y^3 = e \rangle \]
  are isomorphic.
  \begin{solution}
    Note that
    \begin{align*}
      a^bb^a &= (b^{-1}a^2)^2 \\
      \iff b^{-1}aba^{-1}ba &= b^{-1}a^2b^{-1}a^2 \\
      \iff ba^{-1}b &= ab^{-1}a \\
      \iff ba^{-1}ba^{-1}b &= a \\
      \iff (a^{-1}b)^3 &= e
    \end{align*}
    Therefore
    \begin{align*}
      \langle a,b \mid a^bb^a = (b^{-1}a^2)^2, (b^{-1}a^2)^2 \rangle &\cong \langle a,b \mid a^bb^a = (b^{-1}a^2)^2, (b^{-1}a^2)^2, (a^{-1}b)^3 \rangle \\
                                                                     &\cong \langle a,b \mid (b^{-1}a^2)^2, (a^{-1}b)^3 \rangle.
    \end{align*}
    Next we rewrite in terms of new generators:
    \begin{align*}
      \langle a,b \mid (b^{-1}a^2)^2, (a^{-1}b)^3 \rangle &\cong \langle a,b,x \mid (b^{-1}a^2)^2, (a^{-1}b)^3, x = b^{-1}a^2 \rangle \\
                                                          &\cong \langle a,b,x \mid (b^{-1}a^2)^2, (a^{-1}b)^3, x = b^{-1}a^2, x^2 \rangle \\
                                                          &\cong \langle a,b,x \mid (a^{-1}b)^3, x=b^{-1}a^2, x^2 \rangle \\
                                                          &\cong \langle a,b,x,y \mid (a^{-1}b)^3, x=b^{-1}a^2, x^2, y=ax^{-1} \rangle \\
                                                          &\cong \langle a,b,x,y \mid (a^{-1}b)^3, x=b^{-1}a^2, x^2, y=ax^{-1}, y^3 \rangle \\
                                                          &\cong \langle a,b,x,y \mid x=b^{-1}a^2, x^2, y=ax^{-1}, y^3 \rangle \\
                                                          &\cong \langle a,b,x,y \mid x=b^{-1}a^2, x^2, y=ax^{-1}, y^3, b=a^2x^{-1} \rangle \\
                                                          &\cong \langle a,b,x,y \mid x^2, y=ax^{-1}, y^3, b=a^2x^{-1} \rangle \\
                                                          &\cong \langle a,x,y \mid x^2, y=ax^{-1}, y^3 \rangle \\
                                                          &\cong \langle a,x,y \mid x^2, y=ax^{-1}, y^3, a=yx \rangle \\
                                                          &\cong \langle a,x,y \mid x^2, y^3, a=yx \rangle \\
                                                          &\cong \langle x,y \mid x^2, y^3 \rangle.
    \end{align*}
    Therefore $\langle a,b \mid a^bb^a = (b^{-1}a^2)^2 = e \rangle\cong\langle x,y \mid x^2, y^3 \rangle$.
  \end{solution}

\question Do you recognize the group
  \[ \langle a,b,c \mid a^3=b^2=c^2=(ab)^2=(bc)^2=[a,c]=e \rangle\ ? \]
  What is the order of the element $abc$?
  \begin{solution}
    Given $b^2=c^2=e$, we have
    \[ (bc)^2 = e \iff bcbc = e \iff b^{-1}c^{-1}bc = e \iff [b,c]=e. \]
    Therefore, using the result from \textbf{4b} we have
    \begin{align*}
      \langle a,b,c \mid a^3, b^2, c^2, (ab)^2, (bc)^2, [a,c] \rangle &\cong \langle a,b,c \mid a^3, b^2, c^2, (ab)^2, (bc)^2, [a,c], [b,c] \rangle \\
                                                                      &\cong \langle a,b,c \mid a^3, b^2, c^2, (ab)^2, [a,c], [b,c] \rangle \\
                                                                      &\cong \langle a,b \mid a^3, b^2, (ab)^2 \rangle \times \langle c \mid c^2 \rangle \\
                                                                      &\cong D(3,2,2) \times Z_2 \\
                                                                      &\cong D_3 \times Z_2.
    \end{align*}
  \end{solution}

\question Prove the converse of the result in (10) of Example 7:
  \[ (m,n)\neq1 \implies Z_{mn} \not\cong Z_m\times Z_n. \]
  \begin{solution}
    Every element of $Z_m\times Z_n$ has order dividing $[m,n]$, where $[m,n]$ denotes the least common multiple of $m,n$. Since $(m,n)[m,n]=mn$, it follows that $[m,n]<mn$, and hence $Z_m\times Z_n$ has no element of order $mn$. Thus $Z_{mn}\not\cong Z_m\times Z_n$.
  \end{solution}

\question Prove that the groups
  \[ \langle x,y \mid xyx = y, yxy = x \rangle, \quad \langle a,b \mid a^2=b^2, a^{-1}ba=b^{-1} \rangle \]
  are isomorphic.
  \begin{solution}
    Suppose that $xyx=y$ and $yxy=x$. Then $x^2=x(yxy)=(xyx)y=y^2$, and $x^{-1}yx=x^{-1}(yxy)y^{-1}=x^{-1}xy^{-1}=y^{-1}$. Conversely, if $x^2=y^2$ and $x^{-1}yx=y^{-1}$, then $yxy=x(x^{-1}yx)y=xy^{-1}y=x$, and $xyx=x(yxy)y^{-1}=x^2y^{-1}=y^2y^{-1}=y$. Hence $\overline{\{(xyx)y^{-1}, (yxy)x^{-1}\}}=\overline{\{x^2y^{-2}, x^{-1}yxy\}}$, and thus
    \begin{align*}
      \langle x,y \mid xyx=y, yxy=x \rangle &= \langle x,y \mid x^2=y^2, x^{-1}yx=y^{-1} \rangle \\
                                            &\cong \langle a,b \mid a^2=b^2, a^{-1}ba=b^{-1} \rangle.
    \end{align*}
  \end{solution}

\question Show that the groups
  \[ \langle a,b,c,d \mid ab=c, bc=d, cd=a, da=b \rangle \]
  \[ \langle a,b,c,d,f,g \mid ab=d, bc=f, cd=g, df=a, fg=b, ga=c \rangle \]
  are cyclic and find their orders.

\question Consider the group $F(2,6)$ given by
  \[ \langle a,b,c,d,f,g \mid ab=c, bc=d, cd=f, df=g, fg=a, ga=b \rangle. \]
  Prove that there is a homomorphism $\chi\colon F(2,6)\to\Sym(\Z)$ such that
  \[ n(a\chi)=n+1,\ n(b\chi) = -n,\ n\in\Z, \]
  and deduce that $F(2,6)$ is an infinite group.

\question Prove that the group $F(2,5)$ given by
  \[ \langle a,b,c,d,f \mid ab=c, bc=d, cd=f, df=a, fa=b \rangle \]
  is a finite cyclic group.

\question Check the \emph{Witt Identity}
  \[ [[x,y],z^x][[z,x],y^z][[y,z],x^y] = e, \]
  and use it to prove that the group
  \[ \langle x,y,z \mid [x,y]=z, [y,z]=x, [z,x]=y \rangle \]
  is trivial.

\question Let $x$ and $y$ be members of a group such that $x^{-1}y^nx=y^m$, for some $m,n\in\Z$. Prove that, for any $k\in\N$, $x^{-1}y^{n^k}x=y^{m^k}$. Deduce that the group
  \[ \langle x,y \mid y^{-1}x^ny = x^{n+1}, x^{-1}y^nx = y^{n+1} \rangle \]
  is trivial for any $n\in\Z$.

\question Given $G=\langle X\mid R \rangle$, prove that
  \begin{parts}
  \part if $G$ is finitely generated, then there is a finite subset $X'\subseteq X$ such that $G=\langle X' \rangle$,
  \part if $G$ is in addition finitely presented, then there is in addition, a finite subset $R'$ of $F(X')$ such that $G=\langle X' \mid R' \rangle$.
  \end{parts}

\question Solve the word problem for finite groups according to the following scheme:
  \begin{parts}
  \part let $G=\langle X\mid R \rangle$ be a finite presentation of a finite group $G$, and let $w\in F(X)$;
  \part enumerate the elements of $\overline{R}\lhd F(X)$ as products of conjugates of $R^\pm$;
  \part enumerate the homomorphisms: $G\to S_n$, $n\in\N$ as mappings $X\to S_n$ that vanish on $R$, and
  \part by examining alternate entries in the lists (b) and (c), decide in a finite number of steps whether $w=e$ in $G$ or not.
  \end{parts}

\question Let $G$ be the group
  \[ \langle x,y \mid x^2y=y^2x, x^8=e \rangle. \]
  Prove that $y^8=e$ in $G$ and draw the associated van Kampen diagram.

\question Draw a van Kampen diagram for the group
  \[ \langle x,y \mid x^3=y^2, x^3=e, y^{-1}xy=x^2 \rangle, \]
  which shows that the second relation is superfluous.

\question Draw a van Kampen digram which shows that the group
  \[ \langle a,b \mid abab^2 = e = baba^2 \rangle \]
  is abelian.
\end{questions}

%%% Local Variables:
%%% mode: latex
%%% TeX-master: "johnson"
%%% End: