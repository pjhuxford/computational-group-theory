\documentclass[12pt,a4paper]{article}

% Packages

%% Language and font encodings
\usepackage[english]{babel}
\usepackage[utf8]{inputenc}
\usepackage[T1]{fontenc}

%% Layout
\usepackage[a4paper,margin=2.5cm]{geometry}
\usepackage{parskip}

%% Fonts
\usepackage{bm}
\usepackage{microtype}
\usepackage{mathrsfs}
\usepackage{lmodern}

%% Core math
\usepackage{amsmath}
\usepackage{amssymb}
\usepackage{amsthm}
\usepackage{mathtools}

%% Useful math
\usepackage{mathdots}
\usepackage{cool}

%% Algorithms and code
\usepackage{algorithm2e}
\usepackage{listings}
\usepackage{lstlang0}

%% Graphics
\usepackage{xcolor}
\usepackage{graphicx}
\usepackage{tikz}
\usetikzlibrary{arrows,calc,decorations.markings}

%% Linking and numbering
\usepackage{enumitem}
\usepackage[hidelinks]{hyperref}

% Useful commands and definitions

%% Fix spacing between theorems due to parskip

\makeatletter
\def\thm@space@setup{%
  \thm@preskip=\parskip \thm@postskip=0pt
}
\makeatother

%% Listings style

\definecolor{mygreen}{rgb}{0,0.6,0}
\definecolor{mygray}{rgb}{0.5,0.5,0.5}
\definecolor{mymauve}{rgb}{0.58,0,0.82}

\lstset{ %
  backgroundcolor=\color{white},   % choose the background color; you must add \usepackage{color} or \usepackage{xcolor}; should come as last argument
  basicstyle=\footnotesize,        % the size of the fonts that are used for the code
  breakatwhitespace=false,         % sets if automatic breaks should only happen at whitespace
  breaklines=true,                 % sets automatic line breaking
  captionpos=b,                    % sets the caption-position to bottom
  commentstyle=\color{mygreen},    % comment style
  deletekeywords={...},            % if you want to delete keywords from the given language
  escapeinside={\%*}{*)},          % if you want to add LaTeX within your code
  extendedchars=true,              % lets you use non-ASCII characters; for 8-bits encodings only, does not work with UTF-8
  frame=single,	                   % adds a frame around the code
  keepspaces=true,                 % keeps spaces in text, useful for keeping indentation of code (possibly needs columns=flexible)
  keywordstyle=\color{blue},       % keyword style
  language=Octave,                 % the language of the code
  morekeywords={*,...},            % if you want to add more keywords to the set
  numbers=left,                    % where to put the line-numbers; possible values are (none, left, right)
  numbersep=5pt,                   % how far the line-numbers are from the code
  numberstyle=\tiny\color{mygray}, % the style that is used for the line-numbers
  rulecolor=\color{black},         % if not set, the frame-color may be changed on line-breaks within not-black text (e.g. comments (green here))
  showspaces=false,                % show spaces everywhere adding particular underscores; it overrides 'showstringspaces'
  showstringspaces=false,          % underline spaces within strings only
  showtabs=false,                  % show tabs within strings adding particular underscores
  stepnumber=2,                    % the step between two line-numbers. If it's 1, each line will be numbered
  stringstyle=\color{mymauve},     % string literal style
  tabsize=2,	                   % sets default tabsize to 2 spaces
  title=\lstname                   % show the filename of files included with \lstinputlisting; also try caption instead of title
}

%% The standard sets of numbers
\newcommand{\C}{\mathbb{C}}
\newcommand{\N}{\mathbb{N}}
\newcommand{\Q}{\mathbb{Q}}
\newcommand{\R}{\mathbb{R}}
\newcommand{\Z}{\mathbb{Z}}

%% \abs{} and \norm{} give absolute value and norm respectively
%% \abs*{} and \norm*{} give versions which resize
\DeclarePairedDelimiter\abs{\lvert}{\rvert}
\DeclarePairedDelimiter\norm{\lVert}{\rVert}

\DeclareMathOperator{\Mor}{Mor}
\DeclareMathOperator{\Map}{Map}
\DeclareMathOperator{\im}{im}
\DeclareMathOperator{\Aut}{Aut}
\DeclareMathOperator{\Sym}{Sym}
\makeatletter
\newcommand*{\bdiv}{%
  \nonscript\mskip-\medmuskip\mkern5mu%
  \mathbin{\operator@font div}\penalty900\mkern5mu%
  \nonscript\mskip-\medmuskip
}
\makeatother

\theoremstyle{definition}
\newtheorem{theorem}{Theorem}[section]
\newtheorem{definition}[theorem]{Definition}
\newtheorem{corollary}[theorem]{Corollary}
\newtheorem{lemma}[theorem]{Lemma}

\title{Integer Matrix Diagonalization}
\author{Peter Huxford}

\begin{document}

\maketitle

\begin{abstract}
  The aim of this report is to better understand the theory of finitely generated abelian groups, and computational methods pertaining to them. In particular, we shall see that an important idea is the \emph{Smith Normal Form} of an integer matrix.

Naive methods inspired by Gaussian elimination to compute the Smith Normal Form of a matrix can require arbitrary precision arithmetic for relatively innocuous inputs. We will explore some useful modular techniques, which allow us to perform calculations modulo $d$, for an appropriately chosen $d$.
\end{abstract}

\section{Preliminaries}

In these notes, we will always write the group operation of an abelian group additively, unless otherwise stated. 

\begin{definition}[Universal Property]
  A group $G$ is \emph{free abelian} on a subset $X\subseteq G$, if every map from $X$ to an abelian group $H$ extends to a unique homomorphism $G\to H$. We call $X$ a \emph{basis} for $G$ if $G$ is free abelian on $X$.
\end{definition}

There is a similarity with vector spaces: $B$ is a basis of a vector space $V$ iff every map from $B$ to a vector space $W$ extends uniquely to a linear map $V\to W$. The existence of such a linear map is equivalent $B$ being a linearly independent set, while the uniqueness is equivalent to $B$ spanning $V$.


\begin{theorem}
  The free abelian groups with finite basis, up to isomorphism, consist of the groups $\Z^n$ for $n\in\N$.
\end{theorem}

\begin{proof}
  Let $e_i\in\Z^n$ denote the $i$th standard basis vector. Every element of $\Z^n$ has a unique representation $m_1e_1+\cdots m_ne_n$ for integers $m_i\in\Z$. Thus given a map $\{e_1,\ldots,e_n\}$ to a group $H$, we can extend it to a group homomorphism $\Z^n\to H$ as follows. Say $e_i\mapsto h_i$, then define a mapping $\Z^n\to H$ by
  \[ m_1e_1 + \cdots + m_ne_n \mapsto m_1h_1 + \cdots + m_nh_n. \]
  It is readily seen that this is a group homomorphism. Moreover, any group homomorphism out of $\Z^n$ must agree with this. Hence $\Z^n$ is free abelian on $\{e_1,\ldots,e_n\}$.

  Conversely, suppose that $G$ is free abelian on a finite subset $X=\{x_1,\ldots,x_n\}$. By the universal property, we get a homomorphism $\phi\colon G\to\Z^n$ which maps $x_i\mapsto e_i$. Similarly we have a homomorphism $\psi\colon\Z^n\to G$ which maps $e_i\mapsto x_i$. Then $\psi\circ\phi$ is a homomorphism $G\to G$ fixing $X$. By the uniqueness in the universal property, $\psi\circ\phi$ must be the identity on $G$. Similarly $\phi\circ\psi$ is the identity on $\Z^n$. Hence $G\cong\Z^n$.
\end{proof}

Let $G$ be an abelian group generated by $n$ elements. Since $\Z^n$ is free abelian there is a surjective homomorphism $\Z^n\twoheadrightarrow G$ with kernel $H$. By the first isomorphism theorem, $\Z^n/H$ is isomorphic to $G$. Thus understanding what subgroups of $\Z^n$ look like will help us understand the structure of finite abelian groups.

\begin{theorem}[Dedekind]
  A subgroup $H$ of $\Z^n$ can be generated by $\leq n$ elements.
\end{theorem}

\begin{proof}
  We proceed by induction on $n$. Defining $\Z^0\coloneqq\{0\}$, the case $n=0$ becomes trivial. Let $n>0$, and let $\varphi\colon\Z^n\twoheadrightarrow\Z^n/\langle e_n \rangle$ be the natural map, noting that $\Z^n/\langle e_n \rangle\cong\Z^{n-1}$. Suppose $H\leq\Z^n$, then $\varphi(H)$ is isomorphic to a subgroup of $\Z^{n-1}$. Inductively, we may assume $\varphi(H)=\langle h_1 + \langle e_n \rangle, \ldots, h_{n-1} + \langle e_n \rangle \rangle$, for $h_i\in H$. Note that $H\cap\langle e_n \rangle$ is cyclic, so let $H\cap\langle e_n \rangle=\langle h_n \rangle$ for some $h_n\in H$. We claim $H=\langle h_1,\ldots,h_n \rangle$.

  If $h\in H$, then $\varphi(h)=m_1h_1+\cdots+ m_{n-1}h_{n-1}+\langle e_n \rangle$ for some $m_i\in\Z$. Therefore $h-(m_1h_1+\cdots+m_{n-1}h_{n-1})\in H\cap\langle e_n \rangle=\langle h_n \rangle$. Thus $h\in\langle h_1,\ldots,h_n \rangle$. By induction the theorem follows.
\end{proof}

We now know that every finitely generated abelian group $G$ is isomorphic to $\Z^n/H$ for some $H\leq\Z^n$, and also that $H$ can be generated by at most $n$ elements. Say $H=\langle h_1,\ldots,h_m \rangle$, then $H$ consists of all integral linear combinations of the $h_i$. Organising the $h_i$ into rows of a matrix motivates the following definition.

\begin{definition}
  Given an $m\times n$ integer matrix $A$, we define the integer row space $S(A)\leq\Z^n$ to be the collection of all integral linear combinations of rows of $A$.
\end{definition}

What we have seen so far tells us that all finitely generated abelian groups take the form $\Z^n/S(A)$ for some $m\times n$ integer matrix $A$, so understanding the nature of $S(A)$ and the quotient $\Z^n/S(A)$ will help us to understand finitely generated abelian groups.

Note that Dedekind's theorem only provides us with an existence statement. We know that every finitely generated abelian group is isomorphic to $\Z^n/S(A)$ for appropriately chosen $n$ and $A$, but as of yet it is unclear whether the relevant matrix is easy to find. In fact often it is.

For instance, given a finite presentation $G=\langle X \mid R \rangle$, the abelianisation $G_{ab}\coloneqq G/[G,G]$ has presentation $\langle X \mid R, [X,X] \rangle$. If $X=\{x_1,\ldots,x_n\}$, we can rewrite each of the relations in $R$ in the form $x_1^{\alpha_1}\cdots x_n^{\alpha_n}$, $\alpha_i\in\Z$. The homomorphism $\Z^n\twoheadrightarrow G_{ab}$ mapping $e_i\mapsto x_i$ has kernel generated by the $(\alpha_1,\ldots,\alpha_n)$ appearing in the relations. Using these rows we can form a ``relation matrix'' $A$, and then $G_{ab}\cong\Z^n/S(A)$.

Many questions one can ask about finitely presented groups, for instance the problem of deciding if a finitely presented group is infinite, are in general undecidable. However if the abelianisation of a group is infinite, then the original group must be infinite also. Later in this report we will see precisely how questions like this can be answered.

\section{Row Reduction}

In linear algebra, to better understand the structure of the row space of a matrix, we develop the theory of Gaussian elimination. There is a notion of a reduced row echelon form, from which we can more easily determine membership of vectors in the row space.

In order to develop a similar theory for integer row spaces, we must take care in defining the integer row operations. Because we wish for the row space to remain unchanged after an integer row operation, not only must each operation involve only integer combinations of other rows, but we must be able to reverse the effect with an integer row operation.

\begin{definition}
  An \emph{integer row operation} applied to a matrix is one of the following:
  \begin{enumerate}
  \item Swap two rows.
  \item Multiply a row by $-1$.
  \item Add an integer multiple of one row to another row.
  \end{enumerate}
  We say that $m\times n$ integer matrices $A$ and $B$ are \emph{row equivalent} if there is a sequence of integer row operations transforming one into the other, and we write $A\sim B$.
\end{definition}

We only allow multiplication of rows by $-1$, because in order to be able to reverse such an operation, we would need to be multiplying by a unit of $\Z$. The units of $\Z$ are $\{\pm1\}$, and multiplying a row by 1 would leave the matrix unchanged. The proof of the below theorem is omitted, however it is extremely similar to the corresponding fact in linear algebra.

\begin{theorem}
  $A\sim B \iff S(A)=S(B)$.
\end{theorem}

The corresponding notion of reduced row echelon form for integer row spaces is called row Hermite Normal Form (HNF). Because we restrict ourselves to integer row operations, we cannot do as well as well as reduced row echelon form, however we can get fairly close.

\begin{definition}
  An integer $m\times n$ matrix $A$ is in \emph{row Hermite Normal Form (HNF)} if
  \begin{enumerate}
  \item The nonzero rows of $A$ are the first $r$ rows of $A$, for some $r\leq m$.
  \item If $j_i$ is minimal with $A_{i,j_i}$ nonzero, for $1\leq i\leq r$, then $j_1<j_2<\cdots<j_r$.
  \item If $1\leq i\leq r$, then $A_{i,i_j}>0$.
  \item If $1\leq k<i\leq r$, then $0\leq A_{k,j_i}<A_{i,j_i}$.
  \end{enumerate}
\end{definition}

In the above definition, the $A_{i,j_i}$ act like the pivots in reduced row echelon form. Because we only have access to integer row operations, we are not able to ensure that the pivot entries are all 1, and that the entries above pivots are all zero. However we can at least make the pivot entries positive, and everything above them smaller and non-negative.

For example, below is an integer $3\times4$ matrix in row Hermite Normal Form.
\[ A \coloneqq
  \begin{bmatrix}
    2 & 1 & 2 & 3 \\
    0 & 0 & 7 & 5 \\
    0 & 0 & 0 & 9
  \end{bmatrix}.
\]

Suppose we wish to determine whether a given vector $u\in\Z^4$ lies in the row space of $A$. For instance $u=(4,2,-3,10)$. We seek integers $x,y,z$ with $u=xa_1+ya_2+za_3$, where $a_1,a_2,a_3$ are the rows of $A$. Isolating the first column, we see $4=2x$, so $x=2$. Next set $v=u-2a_1=(0,0,-7,4)$. We require $v=ya_2+za_3$. The second column holds no information. The third column tells us that $-7=7y$, so $y=-1$. Set $w=v+a_2=(0,0,0,9)$. We require $w=za_3$. Observing the last column shows $9=9z$, so $z=1$.

Hence $u=2a_1-a_2+a_3$, and so $u\in S(A)$. If at any stage we found an equation that is not solvable for integer $x,y,z$, then we would know that $u\notin S(A)$. We can see that if $A$ is in HNF, it becomes much simpler to answer questions about $S(A)$.

\begin{theorem}
  If $B$ is an $m\times n$ integer matrix, there is a unique $m\times n$ integer matrix $A$ with $S(A)=S(B)$ and $B$ in row Hermite Normal Form.
\end{theorem}

\begin{proof}
  See the below algorithm (taken from Sims)

  \begin{algorithm}[H]
    \SetKwInOut{Procedure}{Procedure}
    \Procedure{ROW\_REDUCE($B$; $A$);}
    \KwIn{An $m\times n$ integer matrix $B$}
    \KwOut{The matrix $A$ in row Hermite normal form with $A\sim B$}

    $A\coloneqq B$; $i\coloneqq1$; $j\coloneqq1$;

    \While{$i\leq m$ and $j\leq n$}{
      \eIf{$A_{k,j}=0$ for $i\leq k\leq m$}{$j\coloneqq j+1$}{
        \While{there exist distinct $k,\ell$ with $i\leq k,\ell\leq m$ and $0<\abs{A_{k,j}}\leq\abs{A_{k,\ell}}$}{
          $q\coloneqq A_{\ell,j} \bdiv A_{k,j}$\;
          Subtract $q$ times row $k$ of $A$ from row $\ell$
        }
        Let $A_{kj}\neq0$ with $i\leq k\leq m$; (this $k$ is unique)\\
        \If{$k\neq i$}{swap rows $i$ and $k$ of $A$}
        \If{$A_{i,j}<0$}{multiply row $i$ of $A$ by $-1$}
        \For{$\ell\coloneqq 1$ to $i-1$}{
          $q\coloneqq A_{\ell,j} \bdiv A_{i,j}$\;
          Subtract $q$ times row $i$ of $A$ from row $\ell$
        }
        $i\coloneqq i+1$; $j\coloneqq j+1$\;
      }
    }
  \end{algorithm}

  It is not hard to see that this algorithm will terminate, and produce a matrix in HNF which is row equivalent to the input matrix.
\end{proof}

Here is an implementation of the above algorithm in magma. Note that there is some freedom in the above algorithm. In particular the choice of the indices $k,\ell$ can be done in a variety of ways. Some authors argue that certain strategies for choosing $k,\ell$ lead to more efficient algorithms in practice.

Below is an implementation of this algorithm below in magma, which uses \emph{the Rosser strategy}. Here we pick $k,\ell$ so that $\abs{A_{\ell,j}}$ is as large as possible, and $\abs{A_{k,j}}$ is as large as possible with $k\neq\ell$. This will cause $q$ to be small, and so we would expect lower modulus entries to be produced throughout.

\lstinputlisting[language=magma,tabsize=8]{../magma/RowReduce.m}

\end{document}
