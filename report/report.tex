\documentclass[12pt,a4paper]{article}

% Packages

%% Language and font encodings
\usepackage[english]{babel}
\usepackage[utf8]{inputenc}
\usepackage[T1]{fontenc}

%% Layout
\usepackage[a4paper,margin=2.5cm]{geometry}
\usepackage{parskip}

%% Fonts
\usepackage{bm}
\usepackage{microtype}
\usepackage{mathrsfs}
\usepackage{lmodern}

%% Core math
\usepackage{amsmath}
\usepackage{amssymb}
\usepackage{amsthm}
\usepackage{mathtools}

%% Useful math
\usepackage{mathdots}
\usepackage{cool}

%% Algorithms and code
\usepackage{algorithm}
\usepackage[noend]{algpseudocode}
\usepackage{listings}
\usepackage{lstlang0}

%% Graphics
\usepackage{xcolor}
\usepackage{graphicx}
\usepackage{tikz}
\usetikzlibrary{arrows,calc,decorations.markings}

%% Linking and numbering
\usepackage{enumitem}
\usepackage[hidelinks]{hyperref}

% Useful commands and definitions

%% Fix spacing between theorems due to parskip

\makeatletter
\def\thm@space@setup{%
  \thm@preskip=\parskip \thm@postskip=0pt
}
\makeatother

%% The standard sets of numbers
\newcommand{\C}{\mathbb{C}}
\newcommand{\N}{\mathbb{N}}
\newcommand{\Q}{\mathbb{Q}}
\newcommand{\R}{\mathbb{R}}
\newcommand{\Z}{\mathbb{Z}}

%% \abs{} and \norm{} give absolute value and norm respectively
%% \abs*{} and \norm*{} give versions which resize
\DeclarePairedDelimiter\abs{\lvert}{\rvert}
\DeclarePairedDelimiter\norm{\lVert}{\rVert}

\DeclareMathOperator{\Mor}{Mor}
\DeclareMathOperator{\Map}{Map}
\DeclareMathOperator{\im}{im}
\DeclareMathOperator{\Aut}{Aut}
\DeclareMathOperator{\Sym}{Sym}

\theoremstyle{definition}
\newtheorem{theorem}{Theorem}[section]
\newtheorem{definition}[theorem]{Definition}
\newtheorem{corollary}[theorem]{Corollary}
\newtheorem{lemma}[theorem]{Lemma}

\title{Integer Matrix Diagonalization}
\author{Peter Huxford}

\begin{document}

\maketitle

\begin{abstract}
  The aim of this report is to better understand the theory of finitely generated abelian groups, and computational methods pertaining to them. In particular, we shall see that an important idea is the \emph{Smith Normal Form} of an integer matrix.

Naive methods inspired by Gaussian elimination to compute the Smith Normal Form of a matrix can require arbitrary precision arithmetic for relatively innocuous inputs. We will explore some useful modular techniques, which allow us to perform calculations modulo $d$, for an appropriately chosen $d$.
\end{abstract}

\section{Preliminaries}

In these notes, we will always write the group operation of an abelian group additively, unless otherwise stated. 

\begin{definition}[Universal Property]
  A group $G$ is \emph{free abelian} on a subset $X\subseteq G$, if every map from $X$ to an abelian group $H$ extends to a unique homomorphism $G\to H$. We call $X$ a \emph{basis} for $G$ if $G$ is free abelian on $X$.
\end{definition}

There is a similarity with vector spaces: $B$ is a basis of a vector space $V$ iff every map from $B$ to a vector space $W$ extends uniquely to a linear map $V\to W$. The existence of such a linear map is equivalent $B$ being a linearly independent set, while the uniqueness is equivalent to $B$ spanning $V$.


\begin{theorem}
  The free abelian groups with finite basis, up to isomorphism, consist of the groups $\Z^n$ for $n\in\N$.
\end{theorem}

\begin{proof}
  Let $e_i\in\Z^n$ denote the $i$th standard basis vector. Every element of $\Z^n$ has a unique representation $m_1e_1+\cdots m_ne_n$ for integers $m_i\in\Z$. Thus given a map $\{e_1,\ldots,e_n\}$ to a group $H$, we can extend it to a group homomorphism $\Z^n\to H$ as follows. Say $e_i\mapsto h_i$, then define a mapping $\Z^n\to H$ by
  \[ m_1e_1 + \cdots + m_ne_n \mapsto m_1h_1 + \cdots + m_nh_n. \]
  It is readily seen that this is a group homomorphism. Moreover, any group homomorphism out of $\Z^n$ must agree with this. Hence $\Z^n$ is free abelian on $\{e_1,\ldots,e_n\}$.

  Conversely, suppose that $G$ is free abelian on a finite subset $X=\{x_1,\ldots,x_n\}$. By the universal property, we get a homomorphism $\phi\colon G\to\Z^n$ which maps $x_i\mapsto e_i$. Similarly we have a homomorphism $\psi\colon\Z^n\to G$ which maps $e_i\mapsto x_i$. Then $\psi\circ\phi$ is a homomorphism $G\to G$ fixing $X$. By the uniqueness in the universal property, $\psi\circ\phi$ must be the identity on $G$. Similarly $\phi\circ\psi$ is the identity on $\Z^n$. Hence $G\cong\Z^n$.
\end{proof}

Let $G$ be an abelian group generated by $n$ elements. Since $\Z^n$ is free abelian there is a surjective homomorphism $\Z^n\twoheadrightarrow G$ with kernel $H$. By the first isomorphism theorem, $\Z^n/H$ is isomorphic to $G$. Thus understanding what subgroups of $\Z^n$ look like will help us understand the structure of finite abelian groups.

\begin{theorem}[Dedekind]
  A subgroup $H$ of $\Z^n$ can be generated by $\leq n$ elements.
\end{theorem}

\begin{proof}
  We proceed by induction on $n$. Defining $\Z^0\coloneqq\{0\}$, the case $n=0$ becomes trivial. Let $n>0$, and let $\varphi\colon\Z^n\twoheadrightarrow\Z^n/\langle e_n \rangle$ be the natural map, noting that $\Z^n/\langle e_n \rangle\cong\Z^{n-1}$. Suppose $H\leq\Z^n$, then $\varphi(H)$ is isomorphic to a subgroup of $\Z^{n-1}$. Inductively, we may assume $\varphi(H)=\langle h_1 + \langle e_n \rangle, \ldots, h_{n-1} + \langle e_n \rangle \rangle$, for $h_i\in H$. Note that $H\cap\langle e_n \rangle$ is cyclic, so let $H\cap\langle e_n \rangle=\langle h_n \rangle$ for some $h_n\in H$. We claim $H=\langle h_1,\ldots,h_n \rangle$.

  If $h\in H$, then $\varphi(h)=m_1h_1+\cdots+ m_{n-1}h_{n-1}+\langle e_n \rangle$ for some $m_i\in\Z$. Therefore $h-(m_1h_1+\cdots+m_{n-1}h_{n-1})\in H\cap\langle e_n \rangle=\langle h_n \rangle$. Thus $h\in\langle h_1,\ldots,h_n \rangle$. By induction the theorem follows.
\end{proof}

We now know that every finitely generated abelian group $G$ is isomorphic to $\Z^n/H$ for some $H\leq\Z^n$, and also that $H$ can be generated by at most $n$ elements. Say $H=\langle h_1,\ldots,h_m \rangle$, then $H$ consists of all integral linear combinations of the $h_i$. Organising the $h_i$ into rows of a matrix motivates the following definition.

\begin{definition}
  Given an $m\times n$ integer matrix $A$, we define the integer row space $S(A)\leq\Z^n$ to be the collection of all integral linear combinations of rows of $A$.
\end{definition}

What we have seen so far tells us that all finitely generated abelian groups take the form $\Z^n/S(A)$ for some $m\times n$ integer matrix $A$, so understanding the nature of $S(A)$ and the quotient $\Z^n/S(A)$ will help us to understand finitely generated abelian groups.

Note that Dedekind's theorem only provides us with an existence statement. We know that every finitely generated abelian group is isomorphic to $\Z^n/S(A)$ for appropriately chosen $n$ and $A$, but as of yet it is unclear whether the relevant matrix is easy to find. In fact often it is.

For instance, given a finite presentation $G=\langle X \mid R \rangle$, the abelianisation $G_{ab}\coloneqq G/[G,G]$ has presentation $\langle X \mid R, [X,X] \rangle$. If $X=\{x_1,\ldots,x_n\}$, we can rewrite each of the relations in $R$ in the form $x_1^{\alpha_1}\cdots x_n^{\alpha_n}$, $\alpha_i\in\Z$. The homomorphism $\Z^n\twoheadrightarrow G_{ab}$ mapping $e_i\mapsto x_i$ has kernel generated by the $(\alpha_1,\ldots,\alpha_n)$ appearing in the relations. Using these rows we can form a ``relation matrix'' $A$, and then $G_{ab}\cong\Z^n/S(A)$.

Many questions one can ask about finitely presented groups, for instance the problem of deciding if a finitely presented group is infinite, are in general undecidable. However if the abelianisation of a group is infinite, then the original group must be infinite also. Later in this report we will see precisely how questions like this can be answered.

\end{document}
