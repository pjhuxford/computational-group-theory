\documentclass[12pt,a4paper]{article}

% Packages

%% Language and font encodings
\usepackage[english]{babel}
\usepackage[utf8]{inputenc}
\usepackage[T1]{fontenc}

%% Layout
\usepackage[a4paper,margin=2.5cm]{geometry}
\usepackage{parskip}

%% Fonts
\usepackage{bm}
\usepackage{microtype}
\usepackage{mathrsfs}
\usepackage{lmodern}

%% Core math
\usepackage{amsmath}
\usepackage{amssymb}
\usepackage{amsthm}
\usepackage{mathtools}

%% Useful math
\usepackage{mathdots}
\usepackage{cool}

%% Algorithms and code
\usepackage{algorithm2e}
\usepackage{listings}
\usepackage{lstlang0}

%% Graphics
\usepackage{xcolor}
\usepackage{graphicx}
\usepackage{tikz}
\usetikzlibrary{arrows,calc,decorations.markings}

%% Linking and numbering
\usepackage{enumitem}
\usepackage[hidelinks]{hyperref}

%% References
\usepackage[
backend=bibtex,
style=alphabetic,
sorting=ynt
]{biblatex}

\addbibresource{report.bib}

% Useful commands and definitions

%% Fix spacing between theorems due to parskip

\makeatletter
\def\thm@space@setup{%
  \thm@preskip=\parskip \thm@postskip=0pt
}
\makeatother

%% Listings style

\definecolor{mygreen}{rgb}{0,0.6,0}
\definecolor{mygray}{rgb}{0.5,0.5,0.5}
\definecolor{mymauve}{rgb}{0.58,0,0.82}

\lstset{ %
  backgroundcolor=\color{white},   % choose the background color; you must add \usepackage{color} or \usepackage{xcolor}; should come as last argument
  basicstyle=\footnotesize\ttfamily,            % the size of the fonts that are used for the code
  breakatwhitespace=false,         % sets if automatic breaks should only happen at whitespace
  breaklines=true,                 % sets automatic line breaking
  captionpos=b,                    % sets the caption-position to bottom
  commentstyle=\color{mygreen},    % comment style
  deletekeywords={...},            % if you want to delete keywords from the given language
  escapeinside={\%*}{*)},          % if you want to add LaTeX within your code
  extendedchars=true,              % lets you use non-ASCII characters; for 8-bits encodings only, does not work with UTF-8
  frame=single,	                   % adds a frame around the code
  keepspaces=true,                 % keeps spaces in text, useful for keeping indentation of code (possibly needs columns=flexible)
  keywordstyle=\color{blue},       % keyword style
  language=Octave,                 % the language of the code
  morekeywords={*,...},            % if you want to add more keywords to the set
  numbers=left,                    % where to put the line-numbers; possible values are (none, left, right)
  numbersep=5pt,                   % how far the line-numbers are from the code
  numberstyle=\tiny\color{mygray}, % the style that is used for the line-numbers
  rulecolor=\color{black},         % if not set, the frame-color may be changed on line-breaks within not-black text (e.g. comments (green here))
  showspaces=false,                % show spaces everywhere adding particular underscores; it overrides 'showstringspaces'
  showstringspaces=false,          % underline spaces within strings only
  showtabs=false,                  % show tabs within strings adding particular underscores
  stepnumber=2,                    % the step between two line-numbers. If it's 1, each line will be numbered
  stringstyle=\color{mymauve},     % string literal style
  tabsize=2,	                   % sets default tabsize to 2 spaces
  title=\lstname                   % show the filename of files included with \lstinputlisting; also try caption instead of title
}

%% The standard sets of numbers
\newcommand{\C}{\mathbb{C}}
\newcommand{\N}{\mathbb{N}}
\newcommand{\Q}{\mathbb{Q}}
\newcommand{\R}{\mathbb{R}}
\newcommand{\Z}{\mathbb{Z}}

%% \abs{} and \norm{} give absolute value and norm respectively
%% \abs*{} and \norm*{} give versions which resize
\DeclarePairedDelimiter\abs{\lvert}{\rvert}
\DeclarePairedDelimiter\norm{\lVert}{\rVert}

\DeclareMathOperator{\Mor}{Mor}
\DeclareMathOperator{\Map}{Map}
\DeclareMathOperator{\im}{im}
\DeclareMathOperator{\Aut}{Aut}
\DeclareMathOperator{\Sym}{Sym}
\makeatletter
\newcommand*{\bdiv}{%
  \nonscript\mskip-\medmuskip\mkern5mu%
  \mathbin{\operator@font div}\penalty900\mkern5mu%
  \nonscript\mskip-\medmuskip
}
\makeatother

\theoremstyle{definition}
\newtheorem{theorem}{Theorem}[section]
\newtheorem{definition}[theorem]{Definition}
\newtheorem{corollary}[theorem]{Corollary}
\newtheorem{lemma}[theorem]{Lemma}

\title{Integer Matrix Diagonalization}
\author{Peter Huxford}

\begin{document}

\maketitle

\begin{abstract}
  The aim of this report is to better understand the theory of finitely generated abelian groups, and computational methods pertaining to them. We shall see that an important idea is the \emph{Smith Normal Form} of an integer matrix.

Naive methods inspired by Gaussian elimination to compute the Smith Normal Form of a matrix can require arbitrary precision arithmetic for small inputs. We will explore some useful modular techniques, which allow us to perform calculations with respect to an appropriate modulus.
\end{abstract}

\section{Preliminaries}

In these notes, we will always write the group operation of an abelian group additively, unless otherwise stated. 

\begin{definition}[Universal Property]
  A group $G$ is \emph{free abelian} on a subset $X\subseteq G$, if every map from $X$ to an abelian group $H$ extends to a unique homomorphism $G\to H$. We call $X$ a \emph{basis} for $G$ if $G$ is free abelian on $X$.
\end{definition}

There is a similarity with vector spaces: $B$ is a basis of a vector space $V$ if and only if every map from $B$ to a vector space $W$ extends uniquely to a linear map $V\to W$. The existence of such a linear map is equivalent to $B$ being a linearly independent set, while the uniqueness is equivalent to $B$ spanning $V$.


\begin{theorem}
  The free abelian groups with finite basis, up to isomorphism, consist of the groups $\Z^n$ for $n\in\N$.
\end{theorem}

\begin{proof}
  Let $e_i\in\Z^n$ be the $i$th standard basis vector. Each element of $\Z^n$ takes the form $m_1e_1+\cdots m_ne_n$, for unique integers $m_i\in\Z$. Given a map from $\{e_1,\ldots,e_n\}$ to a group $H$, we can extend it to a group homomorphism $\Z^n\to H$ as follows. If $e_i\mapsto h_i$, then define
  \[ m_1e_1 + \cdots + m_ne_n \mapsto m_1h_1 + \cdots + m_nh_n. \]
  It is readily seen that this defines a group homomorphism from $\Z^n$ to $H$, sending $e_i\mapsto h_i$. Moreover, any group homomorphism $\Z^n\to H$ which maps $e_i\mapsto h_i$ must agree with this. Hence $\Z^n$ is free abelian on $\{e_1,\ldots,e_n\}$.

  Conversely, suppose that $G$ is free abelian on a finite subset $X=\{x_1,\ldots,x_n\}$. By the universal property, we obtain a homomorphism $\phi\colon G\to\Z^n$ which maps $x_i\mapsto e_i$. Similarly we have a homomorphism $\psi\colon\Z^n\to G$ which maps $e_i\mapsto x_i$. Then $\psi\circ\phi$ is an endomorphism of $G$ fixing $X$. By the uniqueness in the universal property, $\psi\circ\phi$ must be the identity on $G$. Similarly $\phi\circ\psi$ is the identity on $\Z^n$. Hence $G\cong\Z^n$.
\end{proof}

Let $G$ be an abelian group generated by $n$ elements. Since $\Z^n$ is free abelian there is an epimorphism $\Z^n\twoheadrightarrow G$ with some kernel $H$. By the first isomorphism theorem, $\Z^n/H$ is isomorphic to $G$. Thus understanding the structure of subgroups of $\Z^n$ will give us insight into the structure of finitely generated abelian groups.

\begin{theorem}[Dedekind]
  A subgroup of $\Z^n$ can be generated by at most $n$ elements.
\end{theorem}

\begin{proof}
  We proceed by induction on $n$. Defining $\Z^0\coloneqq\{0\}$, the case $n=0$ becomes trivial. Let $n>0$, and let $\varphi\colon\Z^n\twoheadrightarrow\Z^n/\langle e_n \rangle$ be the natural map. Note that $\Z^n/\langle e_n \rangle\cong\Z^{n-1}$. Suppose $H\leq\Z^n$, then $\varphi(H)$ is isomorphic to a subgroup of $\Z^{n-1}$. Inductively, we may assume $\varphi(H)=\langle h_1 + \langle e_n \rangle, \ldots, h_{n-1} + \langle e_n \rangle \rangle$, for $h_i\in H$. Note that $H\cap\langle e_n \rangle$ is cyclic, so let $H\cap\langle e_n \rangle=\langle h_n \rangle$ for some $h_n\in H$. We claim $H=\langle h_1,\ldots,h_n \rangle$.

  If $h\in H$, then $\varphi(h)=h'+\langle e_n \rangle$ for some $h'\in\langle h_1,\ldots,h_{n-1} \rangle$. Now $h-h'\in H\cap\langle e_n \rangle=\langle h_n \rangle$. Thus $h\in\langle h_1,\ldots,h_n \rangle$, and hence $H=\langle h_1,\ldots,h_n \rangle$. By induction the theorem follows.
\end{proof}

If $H=\langle h_1,\ldots,h_m \rangle\leq\Z^n$, then $H$ consists of all integral linear combinations of the $h_i$. Organising the $h_i$ into rows of a matrix motivates the following definition.

\begin{definition}
  Given an $m\times n$ integer matrix $A$, we define the \emph{integer row space} $S(A)\coloneqq\{xA : x\in\Z^m\}$ of $A$ to be the set of all integral linear combinations of rows of $A$.
\end{definition}

Now each finitely generated abelian group is isomorphic to $\Z^n/S(A)$ for some $m\times n$ integer matrix $A$. For most descriptions of finitely generated abelian groups, we can explicitly find such an $A$. For example, consider the abelianisation $G_{ab}\coloneqq G/[G,G]$ of a finitely presented group $G$.

If $G=\langle X \mid R \rangle$ is a finite presentation, then the abelianisation $G_{ab}$ is a finitely generated abelian group, with presentation $\langle X \mid R, [X,X] \rangle$. If $X=\{x_1,\ldots,x_n\}$, then we can rewrite each of the relations in $R$ in the form $x_1^{\alpha_1}\cdots x_n^{\alpha_n}$, $\alpha_i\in\Z$. The epimorphism $\Z^n\twoheadrightarrow G_{ab}$ sending $e_i\mapsto x_i$ has kernel generated by the $(\alpha_1,\ldots,\alpha_n)$ appearing in the relations. Using these rows we can form a ``relation matrix'' $A$, and then $G_{ab}\cong\Z^n/S(A)$.

\section{Integer Row Reduction}

A notion of row space can be defined for matrices over any ring. When working over a field, we can apply row operations to produce a matrix in reduced row echelon form (e.g. using Gaussian elimination). We develop a similar theory for integer matrices below.

\begin{definition}
  An \emph{integer row operation} applied to a matrix is one of the following:
  \begin{enumerate}
  \item Swap two rows.
  \item Multiply a row by $-1$.
  \item Add an integer multiple of one row to another row.
  \end{enumerate}
  Two $m\times n$ integer matrices $A$ and $B$ are \emph{row equivalent} if there is a sequence of integer row operations transforming one into the other, and we write $A\sim B$.
\end{definition}

In the above definition rows may only be multiplied by $-1$, in contrast to row operations over a field, where rows may be multiplied by any non-zero scalar. This is because the units of a field are its non-zero elements, while the units of $\Z$ are just $1$ and $-1$.

We can reverse each integer row operation with another. The first two types are involutions. To reverse adding $q$ times row $i$ to row $j$, for $i\neq j$, we add $-q$ times row $i$ to row $j$. This makes $\sim$ an equivalence relation on integer $m\times n$ matrices. Moreover, integer row operations preserve row spaces.

\begin{theorem}
  Let $A,B$ be integer matrices. Then $A\sim B \implies S(A)=S(B)$.
\end{theorem}

\begin{proof}
  If $B$ is reached from $A$ by a single integer row operation, then the rows of $B$ are contained in $S(A)$, so $S(B)\subseteq S(A)$. When this is the case, we can also reach $A$ from $B$ by a single integer row operation, thus $S(A)\subseteq S(B)$. Therefore $A\sim B\implies S(A)=S(B)$.
\end{proof}

The corresponding notion of reduced row echelon form for integer matrices is row Hermite Normal Form (HNF).

\begin{definition}
  An integer $m\times n$ matrix $A$ is in \emph{row Hermite Normal Form (HNF)} if
  \begin{enumerate}
  \item The nonzero rows of $A$ are the first $r$ rows of $A$, for some $r\leq m$.
  \item If $j_i$ is minimal with $A_{i,j_i}$ nonzero, for $1\leq i\leq r$, then $j_1<j_2<\cdots<j_r$.
  \item If $1\leq i\leq r$, then $A_{i,i_j}>0$.
  \item If $1\leq k<i\leq r$, then $0\leq A_{k,j_i}<A_{i,j_i}$.
  \end{enumerate}
\end{definition}

In the above definition, the entries $A_{i,j_i}$ behave similarly to the pivot entries in reduced row echelon form. Below is an example integer $3\times4$ matrix in row Hermite Normal Form.
\[ A \coloneqq
  \begin{bmatrix}
    2 & 1 & 2 & 3 \\
    0 & 0 & 7 & 5 \\
    0 & 0 & 0 & 9
  \end{bmatrix}.
\]

Suppose we want to determine whether a given vector $u\in\Z^4$ is in $S(A)$. For instance, let $u=(4,2,-3,10)$. We seek integers $x,y,z$ with $u=xa_1+ya_2+za_3$, where $a_1,a_2,a_3$ are the rows of $A$. Isolating the first column, we see $4=2x$, so $x=2$. Next set $v=u-2a_1=(0,0,-7,4)$. We require $v=ya_2+za_3$. The second column holds no information. The third column tells us that $-7=7y$, so $y=-1$. Set $w=v+a_2=(0,0,0,9)$. We require $w=za_3$. Observing the last column shows $9=9z$, so $z=1$.

Hence $u=2a_1-a_2+a_3$, and so $u\in S(A)$. If at any stage we had found an equation that was not solvable for integer $x,y,z$, then we would have instead concluded that $u\notin S(A)$. Clearly, testing membership in $S(A)$ becomes simple when $A$ is in HNF. One can also show that the nonzero rows of $A$ form a basis of $S(A)$ when $A$ is in HNF.

\begin{theorem}
  If $A$ is an $m\times n$ integer matrix, then there is a unique $m\times n$ integer matrix $B$ with $A\sim B$ and $B$ in row Hermite Normal Form.
\end{theorem}

\begin{proof}
  We prove this by induction. The result holds trivially when $m=0$ or $n=0$. Suppose that $m,n\geq1$, and that the result holds for all smaller matrices. If there are two nonzero entries in the first column, say $0<\abs{A_{k,1}}\leq\abs{A_{\ell.1}}$ with $k\neq\ell$, then we can decrease the quantity $\abs{A_{1,1}}+\cdots+\abs{A_{m,1}}$ as follows.

  First multiply rows $k,\ell$ by $-1$ if necessary so that $A_{k,1},A_{\ell,1}>0$. Next, subtract one of row $k$ away from row $\ell$. Since $0\leq A_{\ell,1}-A_{k,1}<A_{\ell,1}$, this will strictly decrease the quantity $\abs{A_{1,1}}+\cdots+\abs{A_{m,1}}$. Hence we may assume that $A$ has at most one nonzero entry in the first column. If all entries in the first column are zero, then $A$ has the block form

  \[ A = \left[
      \begin{array}{c|ccccc}
        0 &&&&& \\
        \vdots &&& A' && \\
        0 &&&&& \\
      \end{array}\right]
  \]

  By induction we can reduce $A'$ to HNF by row operations. If $A'$ is in HNF then so is $A$, so it follows that we can reduce $A$ to HNF by row operations. Suppose then that $A$ has only one nonzero entry in the first column. By swapping rows and multiplying by $-1$ if necessary, we may assume that this is $A_{1,1}$ and $A_{1,1}>0$. Now $A$ has the block form
  \[ A = \left[
      \begin{array}{c|ccccc}
        A_{1,1} & A_{1,2} && \cdots && A_{1,n} \\
        \hline
        0 &&&&& \\
        \vdots &&& A' && \\
        0 &&&&& \\
      \end{array}\right]
  \]
  By induction we can reduce $A'$ to HNF by row operations, so we may assume that $A'$ is in HNF. Suppose that the nonzero rows of $A$ are now the first $r$ rows, and that the first nonzero entry in each row is given by $A_{i,j_i}$, for $1\leq i\leq r$. Let $2\leq k\leq r$, and suppose we have already arranged for $0\leq A_{1,j_i}<A_{i,j_i}$ to hold, for each $i=2,\ldots,k-1$.

  Using the division algorithm we may write $A_{1,j_k}=qA_{k,j_k}+r$, where $0\leq r<A_{k,j_k}$. We then subtract $q$ times row $k$ away from row 1. Because $A_{k,j_k}$ is the first nonzero entry in row $k$, and $1=j_1<j_2<\cdots<j_k$, we still have $0\leq A_{1,j_i}<A_{i,j_i}$ for each $i=2,\ldots,k-1$.

  Hence by induction we can reduce $A$ to some matrix $B$ in HNF, by integer row operations. Let $B'$ be another matrix for which $A\sim B'$ and $B'$ is in HNF. Let $b_1,\ldots,b_m$ and $b_1',\ldots,b_m'$ be the rows of $B$ and $B'$ respectively. If $B\neq B'$, then there are entries with $B_{i,j}\neq B'_{i,j}$. Choose such $i,j$ with $j$ minimal, then without loss of generality $B_{i,j}>B_{i,j}'$. We have $b_i,b_i'\in S(B)=S(A)=S(B')$, hence $b_i-b_i'\in S(B)$.

  Suppose that only the first $r$ rows of $B$ are nonzero, and let $B_{i,j_i}$ be the first nonzero entry in $b_i$ for $1\leq i\leq r$. The first $j-1$ entries of $b_i-b_i'$ are zero, so $b_i-b_i'$ is an integral linear combination of the rows $b_k$ with $j_k\geq j$. However $b_{ij}-b_{ij}'\neq0$, and so we must have $j_k=j$ for some $k$ and $B_{k,j}\mid B_{i,j}-B_{i,j}'$. Since $0\leq B_{i,j}'< B_{i,j}<B_{k,j}$, we must have $\abs{B_{i,j}-B_{i,j}'}<B_{k,j}$, so $B_{i,j}-B_{i,j}'=0$, which is a contradiction. Therefore $B=B'$, and so each integer matrix $A$ is row equivalent to a unique integer matrix $B$ in HNF.
\end{proof}

\begin{corollary}
  Let $A,B$ be $m\times n$ integer matrices. Then $S(A)=S(B)\implies A\sim B$.
\end{corollary}

\begin{proof}
  It suffices to prove that if $A$ and $B$ are in HNF, then $S(A)=S(B)\implies A=B$. We refer the reader to \cite{sims} for the proof.
\end{proof}

The proof of this theorem is readily turned into an procedure, e.g. this one (from \cite{sims})

\begin{algorithm}[H]
  \SetKwInOut{Procedure}{Procedure}
  \Procedure{ROW\_REDUCE(\textasciitilde$A$);}
  \KwIn{An $m\times n$ integer matrix $A$}
  \KwResult{Integer row operations are applied to $A$ to reach row Hermite Normal Form}

  $A\coloneqq B$; $i\coloneqq1$; $j\coloneqq1$;

  \While{$i\leq m$ and $j\leq n$}{
    \eIf{$A_{k,j}=0$ for $i\leq k\leq m$}{$j\coloneqq j+1$}{
      \While{there exist distinct $k,\ell$ with $i\leq k,\ell\leq m$ and $0<\abs{A_{k,j}}\leq\abs{A_{\ell,j}}$}{
        $q\coloneqq A_{\ell,j} \bdiv A_{k,j}$\;
        Subtract $q$ times row $k$ of $A$ from row $\ell$
      }
      Let $A_{kj}\neq0$ with $i\leq k\leq m$; (this $k$ is unique)\\
      \If{$k\neq i$}{swap rows $i$ and $k$ of $A$}
      \If{$A_{i,j}<0$}{multiply row $i$ of $A$ by $-1$}
      \For{$\ell\coloneqq 1$ to $i-1$}{
        $q\coloneqq A_{\ell,j} \bdiv A_{i,j}$\;
        Subtract $q$ times row $i$ of $A$ from row $\ell$
      }
      $i\coloneqq i+1$; $j\coloneqq j+1$\;
    }
  }
\end{algorithm}

Note there is some freedom when implementing the above algorithm. When there are at least two nonzero entries in the $j$th column beyond the $i$th row, we have freedom over which indices $k,\ell\geq i$ are selected with $0<\abs{A_{k,j}}\leq\abs{A_{\ell,j}}$. If $k,\ell$ are selected so that $\abs{A_{\ell,j}}-\abs{A_{k,j}}$ is maximized, then that the quantity $\abs{A_{1,j}}+\cdots+\abs{A_{m,j}}$ decreases as much as possible in each iteration, which one might think would speed up computation.

Below is an implementation of this algorithm in {\sc Magma}, which employs a different strategy as described in \cite{rosser}. Here $k,\ell$ are chosen so that $\abs{A_{\ell,j}}$ is as large as possible, and $\abs{A_{k,j}}$ is as large as possible with $k\neq\ell$. Many authors recommend this because it tends to keep the size of the entries small during the procedure.\newline

\lstinputlisting[language=magma,tabsize=8]{../magma/RowReduce.m}

\end{proof}




\printbibliography

\end{document}
